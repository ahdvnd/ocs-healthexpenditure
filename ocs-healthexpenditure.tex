\documentclass[]{article}
\usepackage{lmodern}
\usepackage{amssymb,amsmath}
\usepackage{ifxetex,ifluatex}
\usepackage{fixltx2e} % provides \textsubscript
\ifnum 0\ifxetex 1\fi\ifluatex 1\fi=0 % if pdftex
  \usepackage[T1]{fontenc}
  \usepackage[utf8]{inputenc}
\else % if luatex or xelatex
  \ifxetex
    \usepackage{mathspec}
  \else
    \usepackage{fontspec}
  \fi
  \defaultfontfeatures{Ligatures=TeX,Scale=MatchLowercase}
\fi
% use upquote if available, for straight quotes in verbatim environments
\IfFileExists{upquote.sty}{\usepackage{upquote}}{}
% use microtype if available
\IfFileExists{microtype.sty}{%
\usepackage{microtype}
\UseMicrotypeSet[protrusion]{basicmath} % disable protrusion for tt fonts
}{}
\usepackage[margin=1in]{geometry}
\usepackage{hyperref}
\hypersetup{unicode=true,
            pdftitle={OpenCaseStudies - Health Expenditure},
            pdfborder={0 0 0},
            breaklinks=true}
\urlstyle{same}  % don't use monospace font for urls
\usepackage{color}
\usepackage{fancyvrb}
\newcommand{\VerbBar}{|}
\newcommand{\VERB}{\Verb[commandchars=\\\{\}]}
\DefineVerbatimEnvironment{Highlighting}{Verbatim}{commandchars=\\\{\}}
% Add ',fontsize=\small' for more characters per line
\usepackage{framed}
\definecolor{shadecolor}{RGB}{248,248,248}
\newenvironment{Shaded}{\begin{snugshade}}{\end{snugshade}}
\newcommand{\AlertTok}[1]{\textcolor[rgb]{0.94,0.16,0.16}{#1}}
\newcommand{\AnnotationTok}[1]{\textcolor[rgb]{0.56,0.35,0.01}{\textbf{\textit{#1}}}}
\newcommand{\AttributeTok}[1]{\textcolor[rgb]{0.77,0.63,0.00}{#1}}
\newcommand{\BaseNTok}[1]{\textcolor[rgb]{0.00,0.00,0.81}{#1}}
\newcommand{\BuiltInTok}[1]{#1}
\newcommand{\CharTok}[1]{\textcolor[rgb]{0.31,0.60,0.02}{#1}}
\newcommand{\CommentTok}[1]{\textcolor[rgb]{0.56,0.35,0.01}{\textit{#1}}}
\newcommand{\CommentVarTok}[1]{\textcolor[rgb]{0.56,0.35,0.01}{\textbf{\textit{#1}}}}
\newcommand{\ConstantTok}[1]{\textcolor[rgb]{0.00,0.00,0.00}{#1}}
\newcommand{\ControlFlowTok}[1]{\textcolor[rgb]{0.13,0.29,0.53}{\textbf{#1}}}
\newcommand{\DataTypeTok}[1]{\textcolor[rgb]{0.13,0.29,0.53}{#1}}
\newcommand{\DecValTok}[1]{\textcolor[rgb]{0.00,0.00,0.81}{#1}}
\newcommand{\DocumentationTok}[1]{\textcolor[rgb]{0.56,0.35,0.01}{\textbf{\textit{#1}}}}
\newcommand{\ErrorTok}[1]{\textcolor[rgb]{0.64,0.00,0.00}{\textbf{#1}}}
\newcommand{\ExtensionTok}[1]{#1}
\newcommand{\FloatTok}[1]{\textcolor[rgb]{0.00,0.00,0.81}{#1}}
\newcommand{\FunctionTok}[1]{\textcolor[rgb]{0.00,0.00,0.00}{#1}}
\newcommand{\ImportTok}[1]{#1}
\newcommand{\InformationTok}[1]{\textcolor[rgb]{0.56,0.35,0.01}{\textbf{\textit{#1}}}}
\newcommand{\KeywordTok}[1]{\textcolor[rgb]{0.13,0.29,0.53}{\textbf{#1}}}
\newcommand{\NormalTok}[1]{#1}
\newcommand{\OperatorTok}[1]{\textcolor[rgb]{0.81,0.36,0.00}{\textbf{#1}}}
\newcommand{\OtherTok}[1]{\textcolor[rgb]{0.56,0.35,0.01}{#1}}
\newcommand{\PreprocessorTok}[1]{\textcolor[rgb]{0.56,0.35,0.01}{\textit{#1}}}
\newcommand{\RegionMarkerTok}[1]{#1}
\newcommand{\SpecialCharTok}[1]{\textcolor[rgb]{0.00,0.00,0.00}{#1}}
\newcommand{\SpecialStringTok}[1]{\textcolor[rgb]{0.31,0.60,0.02}{#1}}
\newcommand{\StringTok}[1]{\textcolor[rgb]{0.31,0.60,0.02}{#1}}
\newcommand{\VariableTok}[1]{\textcolor[rgb]{0.00,0.00,0.00}{#1}}
\newcommand{\VerbatimStringTok}[1]{\textcolor[rgb]{0.31,0.60,0.02}{#1}}
\newcommand{\WarningTok}[1]{\textcolor[rgb]{0.56,0.35,0.01}{\textbf{\textit{#1}}}}
\usepackage{longtable,booktabs}
\usepackage{graphicx,grffile}
\makeatletter
\def\maxwidth{\ifdim\Gin@nat@width>\linewidth\linewidth\else\Gin@nat@width\fi}
\def\maxheight{\ifdim\Gin@nat@height>\textheight\textheight\else\Gin@nat@height\fi}
\makeatother
% Scale images if necessary, so that they will not overflow the page
% margins by default, and it is still possible to overwrite the defaults
% using explicit options in \includegraphics[width, height, ...]{}
\setkeys{Gin}{width=\maxwidth,height=\maxheight,keepaspectratio}
\IfFileExists{parskip.sty}{%
\usepackage{parskip}
}{% else
\setlength{\parindent}{0pt}
\setlength{\parskip}{6pt plus 2pt minus 1pt}
}
\setlength{\emergencystretch}{3em}  % prevent overfull lines
\providecommand{\tightlist}{%
  \setlength{\itemsep}{0pt}\setlength{\parskip}{0pt}}
\setcounter{secnumdepth}{0}
% Redefines (sub)paragraphs to behave more like sections
\ifx\paragraph\undefined\else
\let\oldparagraph\paragraph
\renewcommand{\paragraph}[1]{\oldparagraph{#1}\mbox{}}
\fi
\ifx\subparagraph\undefined\else
\let\oldsubparagraph\subparagraph
\renewcommand{\subparagraph}[1]{\oldsubparagraph{#1}\mbox{}}
\fi

%%% Use protect on footnotes to avoid problems with footnotes in titles
\let\rmarkdownfootnote\footnote%
\def\footnote{\protect\rmarkdownfootnote}

%%% Change title format to be more compact
\usepackage{titling}

% Create subtitle command for use in maketitle
\providecommand{\subtitle}[1]{
  \posttitle{
    \begin{center}\large#1\end{center}
    }
}

\setlength{\droptitle}{-2em}

  \title{OpenCaseStudies - Health Expenditure}
    \pretitle{\vspace{\droptitle}\centering\huge}
  \posttitle{\par}
    \author{}
    \preauthor{}\postauthor{}
    \date{}
    \predate{}\postdate{}
  

\begin{document}
\maketitle

{
\setcounter{tocdepth}{2}
\tableofcontents
}
\includegraphics[width=1\linewidth]{./img/SummaryPlot}

\hypertarget{motivation}{%
\section{Motivation}\label{motivation}}

Health policy in the States is complicated, and several forms of
healthcare coverage existed in the United States of America, including
both federal government-led healthcare policy, and private insurance
company. Before making any inference about the relationship between
health condition and health policy, it is important for us to have a
general idea about healthcare economics in the States. Thus, we are
interested in getting sense of the health expenditure, including
healthcare coverage and healthcare spending, across States. More
specifically, the questions are:

\begin{enumerate}
\def\labelenumi{\arabic{enumi}.}
\tightlist
\item
  Is there a relationship between healthcare coverage and healthcare
  spending in the United States?\\
\item
  How does the spending distribution change across geographic regions in
  the United States?\\
\item
  Does the relationship between healthcare coverage and healthcare
  spending in the United States change from 2013 to 2014?
\end{enumerate}

In this case study, we'll walk you through collecting data, importing
data, cleaning data, wrangling data, and visualizing the data, using
well-established and commonly used packages, including
\texttt{datasets}, \texttt{tidyr}, \texttt{dplyr}, \texttt{ggplot2}, and
\texttt{ggrepel}.

\hypertarget{what-is-the-data}{%
\section{What is the data?}\label{what-is-the-data}}

\includegraphics[width=0.9\linewidth]{https://aspe.hhs.gov/system/files/images-reports-basic/70441/fig1}
\href{https://aspe.hhs.gov/basic-report/overview-uninsured-united-states-summary-2011-current-population-survey}{Image
source from US Department of Health and Human Services}

\hypertarget{healthcare-data}{%
\subsection{Healthcare data}\label{healthcare-data}}

We will be using the data from the \href{https://www.kff.org}{Henry J
Kaiser Family Foundation (KFF)}.

\begin{itemize}
\tightlist
\item
  \href{https://www.kff.org/other/state-indicator/total-population/}{Health
  Insurance Coverage of the Total Population} - Includes years 2013-2016
\item
  \href{https://www.kff.org/other/state-indicator/health-care-expenditures-by-state-of-residence-in-millions/}{Health
  Care Expenditures by State of Residence (in millions)} - Includes
  years 1991-2014
\end{itemize}

We have downloaded, re-named and saved these files in the
\href{https://github.com/opencasestudies/ocs-healthexpenditure}{GitHub
repository} under the \texttt{data/KFF/} directory.

Now, before we dig into the data analysis, we need to introduce a set of
R packages that we will use to analyze the data.

\hypertarget{data-import}{%
\section{Data Import}\label{data-import}}

\hypertarget{introduction-to-tidy-data}{%
\subsection{Introduction to ``Tidy
data''}\label{introduction-to-tidy-data}}

The \href{https://www.tidyverse.org}{tidyverse} is \emph{``an
opinionated collection of R packages designed for data science. All
packages share an underlying philosophy and common APIs.''}

Another way of putting it is that it's a set of packages that are useful
specifically for data manipulation, exploration and visualization with a
common philosophy.

\hypertarget{what-is-this-common-philosophy}{%
\paragraph{What is this common
philosophy?}\label{what-is-this-common-philosophy}}

The common philosophy is called \emph{``tidy''} data. It is a standard
way of mapping the meaning of a dataset to its structure.

In \emph{tidy} data:

\begin{itemize}
\tightlist
\item
  Each variable forms a column.
\item
  Each observation forms a row.
\item
  Each type of observational unit forms a table.
\end{itemize}

\includegraphics[width=0.95\linewidth]{http://r4ds.had.co.nz/images/tidy-1}

Below, we are interested in transforming the table on the right to the
the table on the left, which is considered ``tidy''.

\includegraphics[width=0.95\linewidth]{http://r4ds.had.co.nz/images/tidy-9}

Working with tidy data is useful because it creates a structured way of
organizing data values within a data set. This makes the data analysis
process more efficient and simplifies the development of data analysis
tools that work together. In this way, you can focus on the problem you
are investigating, rather than the uninteresting logistics of data.

\hypertarget{what-is-in-the-tidyverse}{%
\subsubsection{\texorpdfstring{1. What is in the
\texttt{tidyverse}?}{1. What is in the tidyverse?}}\label{what-is-in-the-tidyverse}}

We can install and load the set of R packages using
\texttt{install.packages("tidyverse")} function.

When we load the tidyverse package using \texttt{library(tidyverse)},
there are six core R packages that load:

\begin{itemize}
\tightlist
\item
  \href{http://readr.tidyverse.org}{readr}, for data import.
\item
  \href{http://tidyr.tidyverse.org}{tidyr}, for data tidying.
\item
  \href{http://dplyr.tidyverse.org}{dplyr}, for data wrangling.
\item
  \href{http://ggplot2.tidyverse.org}{ggplot2}, for data visualisation.
\item
  \href{http://purrr.tidyverse.org}{purrr}, for functional programming.
\item
  \href{http://tibble.tidyverse.org}{tibble}, for tibbles, a modern
  re-imagining of data frames.
\end{itemize}

Here, we load in the tidyverse.

\begin{Shaded}
\begin{Highlighting}[]
\KeywordTok{library}\NormalTok{(tidyverse)}
\end{Highlighting}
\end{Shaded}

These packages are highlighted in bold here:

\includegraphics[width=0.95\linewidth]{https://rviews.rstudio.com/post/2017-06-09-What-is-the-tidyverse_files/tidyverse1}

Because these packages all share the ``tidy'' philosophy, the data
analysis workflow is easier as you move from package to package.

Here, we will focus on the \texttt{readr}, \texttt{tidyr} and
\texttt{dplyr} R packages to import data, to transform data to the
``tidy'' format, and to wrangle data.

Next, we will give a brief description of the features in each of these
packages.

There are several base R functions that allow you read in data into R,
which you may be familiar with such as \texttt{read.table()},
\texttt{read.csv()}, and \texttt{read.delim()}. Instead of using these,
we will use the functions in the
\href{https://readr.tidyverse.org/articles/readr.html}{readr} R package.
The main reasons for this are

\begin{enumerate}
\def\labelenumi{\arabic{enumi}.}
\tightlist
\item
  Compared to equivalent base R functions, the functions in
  \texttt{readr} are around 10x faster.
\item
  You can specify the column types (e.g character, integer, double,
  logical, date, time, etc)
\item
  All parsing problems are recorded in a data frame.
\end{enumerate}

\hypertarget{read-data-using-the-readr-r-package}{%
\subsection{\texorpdfstring{Read data using the \texttt{readr} R
package}{Read data using the readr R package}}\label{read-data-using-the-readr-r-package}}

\begin{Shaded}
\begin{Highlighting}[]
\KeywordTok{library}\NormalTok{(readr)}
\end{Highlighting}
\end{Shaded}

The main functions in \texttt{readr} are:

\begin{longtable}[]{@{}ll@{}}
\toprule
\begin{minipage}[b]{0.03\columnwidth}\raggedright
\texttt{readr} functions\strut
\end{minipage} & \begin{minipage}[b]{0.91\columnwidth}\raggedright
Description\strut
\end{minipage}\tabularnewline
\midrule
\endhead
\begin{minipage}[t]{0.03\columnwidth}\raggedright
\texttt{read\_delim()}\strut
\end{minipage} & \begin{minipage}[t]{0.91\columnwidth}\raggedright
reads in a flat file data with a given character to separate
fields\strut
\end{minipage}\tabularnewline
\begin{minipage}[t]{0.03\columnwidth}\raggedright
\texttt{read\_csv()}\strut
\end{minipage} & \begin{minipage}[t]{0.91\columnwidth}\raggedright
reads in a CSV file\strut
\end{minipage}\tabularnewline
\begin{minipage}[t]{0.03\columnwidth}\raggedright
\texttt{read\_tsv()}\strut
\end{minipage} & \begin{minipage}[t]{0.91\columnwidth}\raggedright
reads in a file with values separated by tabs\strut
\end{minipage}\tabularnewline
\begin{minipage}[t]{0.03\columnwidth}\raggedright
\texttt{read\_lines()}\strut
\end{minipage} & \begin{minipage}[t]{0.91\columnwidth}\raggedright
reads only a certain number of lines from the file\strut
\end{minipage}\tabularnewline
\begin{minipage}[t]{0.03\columnwidth}\raggedright
\texttt{read\_file()}\strut
\end{minipage} & \begin{minipage}[t]{0.91\columnwidth}\raggedright
reads a complete file into a string\strut
\end{minipage}\tabularnewline
\begin{minipage}[t]{0.03\columnwidth}\raggedright
\texttt{write\_csv()}\strut
\end{minipage} & \begin{minipage}[t]{0.91\columnwidth}\raggedright
writes data frame to CSV\strut
\end{minipage}\tabularnewline
\bottomrule
\end{longtable}

A useful cheatsheet for the functions in the \texttt{readr} package can
be found on RStudio's website:

\includegraphics{https://www.rstudio.com/wp-content/uploads/2018/08/data-import.png}

\hypertarget{read-in-data}{%
\subsubsection{1. Read in data}\label{read-in-data}}

\hypertarget{read-in-health-healthcare-coverage-data}{%
\paragraph{Read in health healthcare coverage
data}\label{read-in-health-healthcare-coverage-data}}

Let's try reading in some data. We will begin by reading in the
\texttt{healthcare-coverage.csv} data.

If we want to see what the header of the file looks like, we can use the
\texttt{read\_lines()} function to peak at the first few lines.

\begin{Shaded}
\begin{Highlighting}[]
\KeywordTok{read_lines}\NormalTok{(}\DataTypeTok{file =} \StringTok{"./data/KFF/healthcare-coverage.csv"}\NormalTok{, }\DataTypeTok{n_max =} \DecValTok{10}\NormalTok{)}
\end{Highlighting}
\end{Shaded}

\begin{verbatim}
 [1] "\"Title: Health Insurance Coverage of the Total Population | The Henry J. Kaiser Family Foundation\""                                                                                                                                                                                                                                                                                                                                                                                                                                                                        
 [2] "\"Timeframe: 2013 - 2016\""                                                                                                                                                                                                                                                                                                                                                                                                                                                                                                                                                  
 [3] "\"Location\",\"2013__Employer\",\"2013__Non-Group\",\"2013__Medicaid\",\"2013__Medicare\",\"2013__Other Public\",\"2013__Uninsured\",\"2013__Total\",\"2014__Employer\",\"2014__Non-Group\",\"2014__Medicaid\",\"2014__Medicare\",\"2014__Other Public\",\"2014__Uninsured\",\"2014__Total\",\"2015__Employer\",\"2015__Non-Group\",\"2015__Medicaid\",\"2015__Medicare\",\"2015__Other Public\",\"2015__Uninsured\",\"2015__Total\",\"2016__Employer\",\"2016__Non-Group\",\"2016__Medicaid\",\"2016__Medicare\",\"2016__Other Public\",\"2016__Uninsured\",\"2016__Total\""
 [4] "\"United States\",\"155696900\",\"13816000\",\"54919100\",\"40876300\",\"6295400\",\"41795100\",\"313401200\",\"154347500\",\"19313000\",\"61650400\",\"41896500\",\"5985000\",\"32967500\",\"316159900\",\"155965800\",\"21816500\",\"62384500\",\"43308400\",\"6422300\",\"28965900\",\"318868500\",\"157381500\",\"21884400\",\"62303400\",\"44550200\",\"6192200\",\"28051900\",\"320372000\""                                                                                                                                                                           
 [5] "\"Alabama\",\"2126500\",\"174200\",\"869700\",\"783000\",\"85600\",\"724800\",\"4763900\",\"2202800\",\"288900\",\"891900\",\"718400\",\"143900\",\"522200\",\"4768000\",\"2218000\",\"291500\",\"911400\",\"719100\",\"174600\",\"519400\",\"4833900\",\"2263800\",\"262400\",\"997000\",\"761200\",\"128800\",\"420800\",\"4834100\""                                                                                                                                                                                                                                      
 [6] "\"Alaska\",\"364900\",\"24000\",\"95000\",\"55200\",\"60600\",\"102200\",\"702000\",\"345300\",\"26800\",\"130100\",\"55300\",\"37300\",\"100800\",\"695700\",\"355700\",\"22300\",\"128100\",\"60900\",\"47700\",\"90500\",\"705300\",\"324400\",\"20300\",\"145400\",\"68200\",\"55600\",\"96900\",\"710800\""                                                                                                                                                                                                                                                             
 [7] "\"Arizona\",\"2883800\",\"170800\",\"1346100\",\"842000\",\"N/A\",\"1223000\",\"6603100\",\"2835200\",\"333500\",\"1639400\",\"911100\",\"N/A\",\"827100\",\"6657200\",\"2766500\",\"278400\",\"1711500\",\"949000\",\"189300\",\"844800\",\"6739500\",\"3010700\",\"377000\",\"1468400\",\"1028000\",\"172500\",\"833700\",\"6890200\""                                                                                                                                                                                                                                     
 [8] "\"Arkansas\",\"1128800\",\"155600\",\"600800\",\"515200\",\"67600\",\"436800\",\"2904800\",\"1176500\",\"231700\",\"639200\",\"479400\",\"82000\",\"287200\",\"2896000\",\"1293700\",\"200200\",\"641400\",\"484500\",\"63700\",\"268400\",\"2953000\",\"1290900\",\"252900\",\"618600\",\"490000\",\"67500\",\"225500\",\"2945300\""                                                                                                                                                                                                                                        
 [9] "\"California\",\"17747300\",\"1986400\",\"8344800\",\"3828500\",\"675400\",\"5594100\",\"38176400\",\"17703700\",\"2778800\",\"9618800\",\"4049000\",\"634400\",\"3916700\",\"38701300\",\"17718300\",\"3444200\",\"10138100\",\"4080100\",\"752700\",\"2980600\",\"39113900\",\"18116200\",\"3195400\",\"9853800\",\"4436000\",\"556100\",\"3030800\",\"39188300\""                                                                                                                                                                                                         
[10] "\"Colorado\",\"2852500\",\"426300\",\"697300\",\"549700\",\"118100\",\"654000\",\"5297800\",\"2489400\",\"397900\",\"1053700\",\"619500\",\"214000\",\"602900\",\"5377400\",\"2706000\",\"346900\",\"1036600\",\"708000\",\"148000\",\"475700\",\"5421300\",\"2872600\",\"370000\",\"855800\",\"692400\",\"190100\",\"528400\",\"5509200\""                                                                                                                                                                                                                                  
\end{verbatim}

It looks like the first two lines are descriptive and are not useful. We
will tell R to skip reading these in using the \texttt{skip} argument in
\texttt{read\_csv()}. The third line looks like it contains the column
names and starting on the fourth line is where the data starts.

\begin{Shaded}
\begin{Highlighting}[]
\NormalTok{coverage <-}\StringTok{ }\KeywordTok{read_csv}\NormalTok{(}\StringTok{"./data/KFF/healthcare-coverage.csv"}\NormalTok{, }
                     \DataTypeTok{skip =} \DecValTok{2}\NormalTok{, }\DataTypeTok{col_names =} \OtherTok{TRUE}\NormalTok{)}
\KeywordTok{head}\NormalTok{(coverage)}
\end{Highlighting}
\end{Shaded}

\begin{verbatim}
# A tibble: 6 x 29
  Location `2013__Employer` `2013__Non-Grou~ `2013__Medicaid`
  <chr>               <dbl>            <dbl>            <dbl>
1 United ~        155696900         13816000         54919100
2 Alabama           2126500           174200           869700
3 Alaska             364900            24000            95000
4 Arizona           2883800           170800          1346100
5 Arkansas          1128800           155600           600800
6 Califor~         17747300          1986400          8344800
# ... with 25 more variables: `2013__Medicare` <dbl>, `2013__Other
#   Public` <chr>, `2013__Uninsured` <dbl>, `2013__Total` <dbl>,
#   `2014__Employer` <dbl>, `2014__Non-Group` <dbl>,
#   `2014__Medicaid` <dbl>, `2014__Medicare` <dbl>, `2014__Other
#   Public` <chr>, `2014__Uninsured` <dbl>, `2014__Total` <dbl>,
#   `2015__Employer` <dbl>, `2015__Non-Group` <dbl>,
#   `2015__Medicaid` <dbl>, `2015__Medicare` <dbl>, `2015__Other
#   Public` <chr>, `2015__Uninsured` <dbl>, `2015__Total` <dbl>,
#   `2016__Employer` <dbl>, `2016__Non-Group` <dbl>,
#   `2016__Medicaid` <dbl>, `2016__Medicare` <dbl>, `2016__Other
#   Public` <chr>, `2016__Uninsured` <dbl>, `2016__Total` <dbl>
\end{verbatim}

\begin{Shaded}
\begin{Highlighting}[]
\KeywordTok{tail}\NormalTok{(coverage)}
\end{Highlighting}
\end{Shaded}

\begin{verbatim}
# A tibble: 6 x 29
  Location `2013__Employer` `2013__Non-Grou~ `2013__Medicaid`
  <chr>               <dbl>            <dbl>            <dbl>
1 <NA>                   NA               NA               NA
2 *Uninsu~               NA               NA               NA
3 <NA>                   NA               NA               NA
4 For exa~               NA               NA               NA
5 <NA>                   NA               NA               NA
6 *N/A*: ~               NA               NA               NA
# ... with 25 more variables: `2013__Medicare` <dbl>, `2013__Other
#   Public` <chr>, `2013__Uninsured` <dbl>, `2013__Total` <dbl>,
#   `2014__Employer` <dbl>, `2014__Non-Group` <dbl>,
#   `2014__Medicaid` <dbl>, `2014__Medicare` <dbl>, `2014__Other
#   Public` <chr>, `2014__Uninsured` <dbl>, `2014__Total` <dbl>,
#   `2015__Employer` <dbl>, `2015__Non-Group` <dbl>,
#   `2015__Medicaid` <dbl>, `2015__Medicare` <dbl>, `2015__Other
#   Public` <chr>, `2015__Uninsured` <dbl>, `2015__Total` <dbl>,
#   `2016__Employer` <dbl>, `2016__Non-Group` <dbl>,
#   `2016__Medicaid` <dbl>, `2016__Medicare` <dbl>, `2016__Other
#   Public` <chr>, `2016__Uninsured` <dbl>, `2016__Total` <dbl>
\end{verbatim}

It looks like we now have the right header, but there are a bunch of NAs
in the end of the data frame because most of it isn't useful data.

Let's take a closer look at the last 30 lines

\begin{Shaded}
\begin{Highlighting}[]
\KeywordTok{tail}\NormalTok{(coverage, }\DataTypeTok{n=}\DecValTok{30}\NormalTok{)}
\end{Highlighting}
\end{Shaded}

\begin{verbatim}
# A tibble: 30 x 29
   Location `2013__Employer` `2013__Non-Grou~ `2013__Medicaid`
   <chr>               <dbl>            <dbl>            <dbl>
 1 Washing~          3541600           309000          1026800
 2 West Vi~           841300            42600           382500
 3 Wiscons~          3154500           225300           907600
 4 Wyoming            305900            19500            74200
 5 Notes                  NA               NA               NA
 6 The maj~               NA               NA               NA
 7 <NA>                   NA               NA               NA
 8 "In thi~               NA               NA               NA
 9 <NA>                   NA               NA               NA
10 Data ex~               NA               NA               NA
# ... with 20 more rows, and 25 more variables: `2013__Medicare` <dbl>,
#   `2013__Other Public` <chr>, `2013__Uninsured` <dbl>,
#   `2013__Total` <dbl>, `2014__Employer` <dbl>, `2014__Non-Group` <dbl>,
#   `2014__Medicaid` <dbl>, `2014__Medicare` <dbl>, `2014__Other
#   Public` <chr>, `2014__Uninsured` <dbl>, `2014__Total` <dbl>,
#   `2015__Employer` <dbl>, `2015__Non-Group` <dbl>,
#   `2015__Medicaid` <dbl>, `2015__Medicare` <dbl>, `2015__Other
#   Public` <chr>, `2015__Uninsured` <dbl>, `2015__Total` <dbl>,
#   `2016__Employer` <dbl>, `2016__Non-Group` <dbl>,
#   `2016__Medicaid` <dbl>, `2016__Medicare` <dbl>, `2016__Other
#   Public` <chr>, `2016__Uninsured` <dbl>, `2016__Total` <dbl>
\end{verbatim}

It looks like there is a line with a string \texttt{Notes} in it and
everything below that line should not be read in. We can use the
\texttt{n\_max} argument here.

\begin{Shaded}
\begin{Highlighting}[]
\NormalTok{coverage <-}\StringTok{ }\KeywordTok{read_csv}\NormalTok{(}\StringTok{"./data/KFF/healthcare-coverage.csv"}\NormalTok{, }
                     \DataTypeTok{skip =} \DecValTok{2}\NormalTok{, }\DataTypeTok{col_names =} \OtherTok{TRUE}\NormalTok{)}
\NormalTok{coverage <-}\StringTok{ }\KeywordTok{read_csv}\NormalTok{(}\StringTok{"./data/KFF/healthcare-coverage.csv"}\NormalTok{, }
                     \DataTypeTok{skip =} \DecValTok{2}\NormalTok{, }\DataTypeTok{col_names =} \OtherTok{TRUE}\NormalTok{,}
                     \DataTypeTok{n_max  =} \KeywordTok{which}\NormalTok{(coverage}\OperatorTok{$}\NormalTok{Location }\OperatorTok{==}\StringTok{ "Notes"}\NormalTok{)}\OperatorTok{-}\DecValTok{1}\NormalTok{)}
\KeywordTok{tail}\NormalTok{(coverage)}
\end{Highlighting}
\end{Shaded}

\begin{verbatim}
# A tibble: 6 x 29
  Location `2013__Employer` `2013__Non-Grou~ `2013__Medicaid`
  <chr>               <dbl>            <dbl>            <dbl>
1 Vermont            317700            26200           123400
2 Virginia          4661600           364800           773200
3 Washing~          3541600           309000          1026800
4 West Vi~           841300            42600           382500
5 Wiscons~          3154500           225300           907600
6 Wyoming            305900            19500            74200
# ... with 25 more variables: `2013__Medicare` <dbl>, `2013__Other
#   Public` <chr>, `2013__Uninsured` <dbl>, `2013__Total` <dbl>,
#   `2014__Employer` <dbl>, `2014__Non-Group` <dbl>,
#   `2014__Medicaid` <dbl>, `2014__Medicare` <dbl>, `2014__Other
#   Public` <chr>, `2014__Uninsured` <dbl>, `2014__Total` <dbl>,
#   `2015__Employer` <dbl>, `2015__Non-Group` <dbl>,
#   `2015__Medicaid` <dbl>, `2015__Medicare` <dbl>, `2015__Other
#   Public` <chr>, `2015__Uninsured` <dbl>, `2015__Total` <dbl>,
#   `2016__Employer` <dbl>, `2016__Non-Group` <dbl>,
#   `2016__Medicaid` <dbl>, `2016__Medicare` <dbl>, `2016__Other
#   Public` <chr>, `2016__Uninsured` <dbl>, `2016__Total` <dbl>
\end{verbatim}

That's better!

\hypertarget{read-in-healthcare-spending-data}{%
\paragraph{Read in healthcare spending
data}\label{read-in-healthcare-spending-data}}

Now because we are also going to want to use in
\texttt{healthcare-spending.csv}, let's read it in now.

\begin{Shaded}
\begin{Highlighting}[]
\NormalTok{spending <-}\StringTok{ }\KeywordTok{read_csv}\NormalTok{(}\StringTok{"./data/KFF/healthcare-spending.csv"}\NormalTok{, }
                     \DataTypeTok{skip =} \DecValTok{2}\NormalTok{, }\DataTypeTok{col_names =} \OtherTok{TRUE}\NormalTok{)}
\NormalTok{spending <-}\StringTok{ }\KeywordTok{read_csv}\NormalTok{(}\StringTok{"./data/KFF/healthcare-spending.csv"}\NormalTok{, }
                     \DataTypeTok{skip =} \DecValTok{2}\NormalTok{, }\DataTypeTok{col_names =} \OtherTok{TRUE}\NormalTok{,}
                     \DataTypeTok{n_max  =} \KeywordTok{which}\NormalTok{(spending}\OperatorTok{$}\NormalTok{Location }\OperatorTok{==}\StringTok{ "Notes"}\NormalTok{)}\OperatorTok{-}\DecValTok{1}\NormalTok{)}
\KeywordTok{tail}\NormalTok{(spending)}
\end{Highlighting}
\end{Shaded}

\begin{verbatim}
# A tibble: 6 x 25
  Location `1991__Total He~ `1992__Total He~ `1993__Total He~
  <chr>               <dbl>            <dbl>            <dbl>
1 Vermont              1330             1421             1522
2 Virginia            14829            15599            16634
3 Washing~            12674            13859            14523
4 West Vi~             4672             5159             5550
5 Wiscons~            12694            13669            14636
6 Wyoming              1023             1067             1171
# ... with 21 more variables: `1994__Total Health Spending` <dbl>,
#   `1995__Total Health Spending` <dbl>, `1996__Total Health
#   Spending` <dbl>, `1997__Total Health Spending` <dbl>, `1998__Total
#   Health Spending` <dbl>, `1999__Total Health Spending` <dbl>,
#   `2000__Total Health Spending` <dbl>, `2001__Total Health
#   Spending` <dbl>, `2002__Total Health Spending` <dbl>, `2003__Total
#   Health Spending` <dbl>, `2004__Total Health Spending` <dbl>,
#   `2005__Total Health Spending` <dbl>, `2006__Total Health
#   Spending` <dbl>, `2007__Total Health Spending` <dbl>, `2008__Total
#   Health Spending` <dbl>, `2009__Total Health Spending` <dbl>,
#   `2010__Total Health Spending` <dbl>, `2011__Total Health
#   Spending` <dbl>, `2012__Total Health Spending` <dbl>, `2013__Total
#   Health Spending` <dbl>, `2014__Total Health Spending` <dbl>
\end{verbatim}

\hypertarget{take-a-glimpse-at-your-data}{%
\subsubsection{\texorpdfstring{2. Take a \texttt{glimpse()} at your
data}{2. Take a glimpse() at your data}}\label{take-a-glimpse-at-your-data}}

One last thing in this section. One way to look at our data would be to
use \texttt{head()} or \texttt{tail()}, as we just saw. Another one you
might have heard of is the \texttt{str()} function. One you might not
have heard of is the \texttt{glimpse()} function. It's used for a
special type of object in R called a \texttt{tibble}. Let's read the
help file to learn more.

\begin{Shaded}
\begin{Highlighting}[]
\NormalTok{?tibble}\OperatorTok{::}\NormalTok{tibble}
\end{Highlighting}
\end{Shaded}

It's kind of like \texttt{print()} where it shows you columns running
down the page. Let's try it out. If we look at our data, say the
\texttt{coverage} data frame, we see that it is not \emph{``tidy''}:

\begin{Shaded}
\begin{Highlighting}[]
\KeywordTok{glimpse}\NormalTok{(coverage)}
\end{Highlighting}
\end{Shaded}

\begin{verbatim}
Observations: 52
Variables: 29
$ Location             <chr> "United States", "Alabama", "Alaska", "Ar...
$ `2013__Employer`     <dbl> 155696900, 2126500, 364900, 2883800, 1128...
$ `2013__Non-Group`    <dbl> 13816000, 174200, 24000, 170800, 155600, ...
$ `2013__Medicaid`     <dbl> 54919100, 869700, 95000, 1346100, 600800,...
$ `2013__Medicare`     <dbl> 40876300, 783000, 55200, 842000, 515200, ...
$ `2013__Other Public` <chr> "6295400", "85600", "60600", "N/A", "6760...
$ `2013__Uninsured`    <dbl> 41795100, 724800, 102200, 1223000, 436800...
$ `2013__Total`        <dbl> 313401200, 4763900, 702000, 6603100, 2904...
$ `2014__Employer`     <dbl> 154347500, 2202800, 345300, 2835200, 1176...
$ `2014__Non-Group`    <dbl> 19313000, 288900, 26800, 333500, 231700, ...
$ `2014__Medicaid`     <dbl> 61650400, 891900, 130100, 1639400, 639200...
$ `2014__Medicare`     <dbl> 41896500, 718400, 55300, 911100, 479400, ...
$ `2014__Other Public` <chr> "5985000", "143900", "37300", "N/A", "820...
$ `2014__Uninsured`    <dbl> 32967500, 522200, 100800, 827100, 287200,...
$ `2014__Total`        <dbl> 316159900, 4768000, 695700, 6657200, 2896...
$ `2015__Employer`     <dbl> 155965800, 2218000, 355700, 2766500, 1293...
$ `2015__Non-Group`    <dbl> 21816500, 291500, 22300, 278400, 200200, ...
$ `2015__Medicaid`     <dbl> 62384500, 911400, 128100, 1711500, 641400...
$ `2015__Medicare`     <dbl> 43308400, 719100, 60900, 949000, 484500, ...
$ `2015__Other Public` <chr> "6422300", "174600", "47700", "189300", "...
$ `2015__Uninsured`    <dbl> 28965900, 519400, 90500, 844800, 268400, ...
$ `2015__Total`        <dbl> 318868500, 4833900, 705300, 6739500, 2953...
$ `2016__Employer`     <dbl> 157381500, 2263800, 324400, 3010700, 1290...
$ `2016__Non-Group`    <dbl> 21884400, 262400, 20300, 377000, 252900, ...
$ `2016__Medicaid`     <dbl> 62303400, 997000, 145400, 1468400, 618600...
$ `2016__Medicare`     <dbl> 44550200, 761200, 68200, 1028000, 490000,...
$ `2016__Other Public` <chr> "6192200", "128800", "55600", "172500", "...
$ `2016__Uninsured`    <dbl> 28051900, 420800, 96900, 833700, 225500, ...
$ `2016__Total`        <dbl> 320372000, 4834100, 710800, 6890200, 2945...
\end{verbatim}

\hypertarget{read-the-state-information-using-the-datasets-r-package}{%
\subsection{\texorpdfstring{Read the State information using the
\texttt{datasets} R
package}{Read the State information using the datasets R package}}\label{read-the-state-information-using-the-datasets-r-package}}

Since our goal is to get sense of the health expenditure, including
healthcare coverage and healthcare spending, \textbf{across States}, it
would be nice add some information about each state. Namely, the state
abbreviation and state region (i.e.~north, south, etc).

For this we use the
\href{https://stat.ethz.ch/R-manual/R-devel/library/datasets/html/state.html}{state}
dataset in the \texttt{datasets} R package.

Before we begin, let's look at what states are there:

\begin{Shaded}
\begin{Highlighting}[]
\KeywordTok{unique}\NormalTok{(coverage}\OperatorTok{$}\NormalTok{Location)}
\end{Highlighting}
\end{Shaded}

\begin{verbatim}
 [1] "United States"        "Alabama"              "Alaska"              
 [4] "Arizona"              "Arkansas"             "California"          
 [7] "Colorado"             "Connecticut"          "Delaware"            
[10] "District of Columbia" "Florida"              "Georgia"             
[13] "Hawaii"               "Idaho"                "Illinois"            
[16] "Indiana"              "Iowa"                 "Kansas"              
[19] "Kentucky"             "Louisiana"            "Maine"               
[22] "Maryland"             "Massachusetts"        "Michigan"            
[25] "Minnesota"            "Mississippi"          "Missouri"            
[28] "Montana"              "Nebraska"             "Nevada"              
[31] "New Hampshire"        "New Jersey"           "New Mexico"          
[34] "New York"             "North Carolina"       "North Dakota"        
[37] "Ohio"                 "Oklahoma"             "Oregon"              
[40] "Pennsylvania"         "Rhode Island"         "South Carolina"      
[43] "South Dakota"         "Tennessee"            "Texas"               
[46] "Utah"                 "Vermont"              "Virginia"            
[49] "Washington"           "West Virginia"        "Wisconsin"           
[52] "Wyoming"             
\end{verbatim}

We see there are more than 50 states because ``United States'' and
``District of Columbia'' are both included.

Let's look what states are inside the \texttt{state} dataset.

\begin{Shaded}
\begin{Highlighting}[]
\KeywordTok{library}\NormalTok{(datasets)}
\KeywordTok{data}\NormalTok{(state)}
\KeywordTok{unique}\NormalTok{(state.name)}
\end{Highlighting}
\end{Shaded}

\begin{verbatim}
 [1] "Alabama"        "Alaska"         "Arizona"        "Arkansas"      
 [5] "California"     "Colorado"       "Connecticut"    "Delaware"      
 [9] "Florida"        "Georgia"        "Hawaii"         "Idaho"         
[13] "Illinois"       "Indiana"        "Iowa"           "Kansas"        
[17] "Kentucky"       "Louisiana"      "Maine"          "Maryland"      
[21] "Massachusetts"  "Michigan"       "Minnesota"      "Mississippi"   
[25] "Missouri"       "Montana"        "Nebraska"       "Nevada"        
[29] "New Hampshire"  "New Jersey"     "New Mexico"     "New York"      
[33] "North Carolina" "North Dakota"   "Ohio"           "Oklahoma"      
[37] "Oregon"         "Pennsylvania"   "Rhode Island"   "South Carolina"
[41] "South Dakota"   "Tennessee"      "Texas"          "Utah"          
[45] "Vermont"        "Virginia"       "Washington"     "West Virginia" 
[49] "Wisconsin"      "Wyoming"       
\end{verbatim}

Ah, ok. So let's start by dealing with DC as a special case.

\begin{Shaded}
\begin{Highlighting}[]
\NormalTok{state.abb <-}\StringTok{ }\KeywordTok{c}\NormalTok{(state.abb, }\StringTok{"DC"}\NormalTok{)}
\NormalTok{state.region <-}\StringTok{ }\KeywordTok{as.factor}\NormalTok{(}\KeywordTok{c}\NormalTok{(}\KeywordTok{as.character}\NormalTok{(state.region), }\StringTok{"South"}\NormalTok{))}
\NormalTok{state.name <-}\StringTok{ }\KeywordTok{c}\NormalTok{(state.name, }\StringTok{"District of Columbia"}\NormalTok{)}
\end{Highlighting}
\end{Shaded}

We will deal with the ``United States'' in the next section.

\hypertarget{data-wrangling}{%
\section{Data Wrangling}\label{data-wrangling}}

\hypertarget{what-is-tidy-data}{%
\subsection{What is ``Tidy Data''?}\label{what-is-tidy-data}}

\hypertarget{glance-at-tidy-data}{%
\paragraph{Glance at ``Tidy Data''}\label{glance-at-tidy-data}}

A subset of the data analysis process can be thought about in the
following way:

\includegraphics[width=0.95\linewidth]{http://r4ds.had.co.nz/diagrams/data-science}

where each of these steps needs its own tools and software to complete.

After we import the data into R, if we are going to take advantage of
the \emph{``tidyverse''}, this means we need to \emph{transform} the
data into a form that is \emph{``tidy''}. If you recall, in \emph{tidy}
data:

\begin{itemize}
\tightlist
\item
  Each variable forms a column.
\item
  Each observation forms a row.
\item
  Each type of observational unit forms a table.
\end{itemize}

For example, consider the following dataset:

\includegraphics{https://github.com/datasciencelabs/2016/raw/master/lectures/wrangling/pics/stocks-by-company.png}

Here:

\begin{itemize}
\tightlist
\item
  each row represents one company (row names are companies)
\item
  each column represent one time point
\item
  the stock prices are defined for each row/column pair
\end{itemize}

Alternatively, a data set can be structured in the following way:

\begin{itemize}
\tightlist
\item
  each row represents one time point (but no row names)
\item
  the first column defines the time variable and the last three columns
  contain the stock prices for three companies
\end{itemize}

\includegraphics{https://github.com/datasciencelabs/2016/raw/master/lectures/wrangling/pics/stocks-by-time.png}

In both cases, the data is the same, but the structure is different.
This can be \emph{frustrating} to deal with as an analyst because the
meaning of the values (rows and columns) in the two data sets are
different. Providing a standardized way of organizing values within a
data set would alleviate a major portion of this frustration.

For motivation, a \emph{tidy} version of the stock data we looked at
above looks like this: (we'll learn how the functions work in just a
moment)

\includegraphics{https://github.com/datasciencelabs/2016/raw/master/lectures/wrangling/pics/stocks-tidy.png}

In this ``tidy'' data set, we have three columns representing three
variables (time, company name and stock price). Every row represents
contains one stock price from a particular time and for a specific
company.

If we consider our \texttt{coverage} dataframe, we see it is also not in
a tidy format. Each row contains information about the coverage level by
\texttt{Location} across years and types of coverage.

\begin{Shaded}
\begin{Highlighting}[]
\NormalTok{coverage[}\DecValTok{1}\OperatorTok{:}\DecValTok{5}\NormalTok{, }\DecValTok{1}\OperatorTok{:}\DecValTok{5}\NormalTok{]}
\end{Highlighting}
\end{Shaded}

\begin{verbatim}
# A tibble: 5 x 5
  Location `2013__Employer` `2013__Non-Grou~ `2013__Medicaid`
  <chr>               <dbl>            <dbl>            <dbl>
1 United ~        155696900         13816000         54919100
2 Alabama           2126500           174200           869700
3 Alaska             364900            24000            95000
4 Arizona           2883800           170800          1346100
5 Arkansas          1128800           155600           600800
# ... with 1 more variable: `2013__Medicare` <dbl>
\end{verbatim}

Now, let's use the \texttt{tidyr} R package to transform our data into a
\emph{tidy} format.

\hypertarget{the-tidyr-r-package}{%
\subsection{\texorpdfstring{The \texttt{tidyr} R
package}{The tidyr R package}}\label{the-tidyr-r-package}}

\hypertarget{what-is-the-tidyr-r-package}{%
\subsubsection{\texorpdfstring{1. What is the \texttt{tidyr} R package
?}{1. What is the tidyr R package ?}}\label{what-is-the-tidyr-r-package}}

\href{https://cran.r-project.org/web/packages/tidyr/vignettes/tidy-data.html}{\texttt{tidyr}}
is an R package that transforms data sets to a tidy format.

This package is installed and loaded when you load the
\texttt{tidyverse} using \texttt{library(tidyverse)}. However, you can
also just load the library by itself.

\begin{Shaded}
\begin{Highlighting}[]
\KeywordTok{library}\NormalTok{(tidyr)}
\end{Highlighting}
\end{Shaded}

The main functions in \texttt{tidyr} are:

\begin{longtable}[]{@{}ll@{}}
\toprule
\begin{minipage}[b]{0.03\columnwidth}\raggedright
\texttt{tidyr} functions\strut
\end{minipage} & \begin{minipage}[b]{0.91\columnwidth}\raggedright
Description\strut
\end{minipage}\tabularnewline
\midrule
\endhead
\begin{minipage}[t]{0.03\columnwidth}\raggedright
\texttt{gather()}\strut
\end{minipage} & \begin{minipage}[t]{0.91\columnwidth}\raggedright
takes multiple columns, and gathers them into key-value pairs, making
``wide'' data longer\strut
\end{minipage}\tabularnewline
\begin{minipage}[t]{0.03\columnwidth}\raggedright
\texttt{separate()}\strut
\end{minipage} & \begin{minipage}[t]{0.91\columnwidth}\raggedright
turns a single character column into multiple columns, making ``long''
data wider\strut
\end{minipage}\tabularnewline
\begin{minipage}[t]{0.03\columnwidth}\raggedright
\texttt{spread()}\strut
\end{minipage} & \begin{minipage}[t]{0.91\columnwidth}\raggedright
spread rows into multiple columns, transforming ``long'' data into
``wide'' format\strut
\end{minipage}\tabularnewline
\bottomrule
\end{longtable}

We'll explore what it means to go between a ``wide'' and ``long'' data
format using \texttt{gather()} , \texttt{separate()}, and
\texttt{spread()}.

A
\href{https://www.rstudio.com/wp-content/uploads/2015/02/data-wrangling-cheatsheet.pdf}{\texttt{tidyr}
cheatsheet} for the functions in the \texttt{tidyr} package can be found
on RStudio's website:

\hypertarget{convert-data-from-wide-format-to-long-format-using-gather}{%
\subsubsection{\texorpdfstring{2. Convert data from wide format to long
format using
\texttt{gather()}}{2. Convert data from wide format to long format using gather()}}\label{convert-data-from-wide-format-to-long-format-using-gather}}

Let's start by looking at the \texttt{gather()} help file

\begin{Shaded}
\begin{Highlighting}[]
\NormalTok{?gather}
\end{Highlighting}
\end{Shaded}

This function gathers multiple columns and collapses them into new
\emph{key-value} pairs. This transform data from \emph{wide} format into
a \emph{long} format.

\begin{itemize}
\tightlist
\item
  The \texttt{key} is the name of the \emph{new} column that you are
  creating which contains the values of the column headings that you are
  gathering
\item
  The \texttt{value} is the name of the \emph{new} column that will
  contain the values themselves
\item
  The third argument defines the columns to gather
\end{itemize}

For example, here we create a column titled \texttt{year\_type} and
\texttt{coverage}. We also want to keep the \texttt{Location} column as
it is because it also contains observational level data.

\begin{Shaded}
\begin{Highlighting}[]
\NormalTok{coverage <-}\StringTok{ }\KeywordTok{gather}\NormalTok{(coverage, }\StringTok{"year_type"}\NormalTok{, }\StringTok{"tot_coverage"}\NormalTok{, }\OperatorTok{-}\NormalTok{Location)}
\NormalTok{coverage}
\end{Highlighting}
\end{Shaded}

\begin{verbatim}
# A tibble: 1,456 x 3
   Location             year_type      tot_coverage
   <chr>                <chr>          <chr>       
 1 United States        2013__Employer 155696900   
 2 Alabama              2013__Employer 2126500     
 3 Alaska               2013__Employer 364900      
 4 Arizona              2013__Employer 2883800     
 5 Arkansas             2013__Employer 1128800     
 6 California           2013__Employer 17747300    
 7 Colorado             2013__Employer 2852500     
 8 Connecticut          2013__Employer 2030500     
 9 Delaware             2013__Employer 473700      
10 District of Columbia 2013__Employer 324300      
# ... with 1,446 more rows
\end{verbatim}

Now we see each row contains one observation. Namely, a
\texttt{Location}, a \texttt{year\_type} and \texttt{coverage}. It would
be nice to separate out the information in the \texttt{year\_type}
column into two columns. We can implement same techniques to the
healthcare spending dataset.

\hypertarget{convert-healthcare-spending-data-to-a-long-format-tidy-format}{%
\paragraph{Convert healthcare spending data to a long format (tidy
format)}\label{convert-healthcare-spending-data-to-a-long-format-tidy-format}}

Let's do the same for the \texttt{spending} data. In this case I will
use \texttt{year} and \texttt{spending} for the \texttt{key} and
\texttt{value}. We also want to keep \texttt{Location} like before.

\begin{Shaded}
\begin{Highlighting}[]
\NormalTok{spending <-}\StringTok{ }\KeywordTok{gather}\NormalTok{(spending, }\StringTok{"year"}\NormalTok{, }\StringTok{"tot_spending"}\NormalTok{, }\OperatorTok{-}\NormalTok{Location)}
\NormalTok{spending}
\end{Highlighting}
\end{Shaded}

\begin{verbatim}
# A tibble: 1,248 x 3
   Location             year                        tot_spending
   <chr>                <chr>                              <dbl>
 1 United States        1991__Total Health Spending       675896
 2 Alabama              1991__Total Health Spending        10393
 3 Alaska               1991__Total Health Spending         1458
 4 Arizona              1991__Total Health Spending         9269
 5 Arkansas             1991__Total Health Spending         5632
 6 California           1991__Total Health Spending        81438
 7 Colorado             1991__Total Health Spending         8460
 8 Connecticut          1991__Total Health Spending        10950
 9 Delaware             1991__Total Health Spending         1938
10 District of Columbia 1991__Total Health Spending         2800
# ... with 1,238 more rows
\end{verbatim}

We will explore how to do that in the Data Wrangling section below. For
now let's learn more about the \texttt{tidyr} package.

\hypertarget{convert-data-from-long-format-to-wide-format-using-spread}{%
\subsubsection{\texorpdfstring{3. Convert data from long format to wide
format using
\texttt{spread()}}{3. Convert data from long format to wide format using spread()}}\label{convert-data-from-long-format-to-wide-format-using-spread}}

In contrast to \emph{gathering} multiple columns into key-value pairs,
we can \emph{spread} a key-value pair across multiple columns.

The function \texttt{spread()} does just that. It transforms data from a
\emph{long} format into a \emph{wide} format.

\begin{itemize}
\tightlist
\item
  The \texttt{key} is the name of the column in your data set that
  contains the values of the column headings that you are spreading
  across multiple columns
\item
  The \texttt{value} is the name of the column that contains the values
  for the multiple columns
\end{itemize}

\begin{Shaded}
\begin{Highlighting}[]
\KeywordTok{spread}\NormalTok{(coverage, year_type, tot_coverage)}
\end{Highlighting}
\end{Shaded}

\begin{verbatim}
# A tibble: 52 x 29
   Location `2013__Employer` `2013__Medicaid` `2013__Medicare`
   <chr>    <chr>            <chr>            <chr>           
 1 Alabama  2126500          869700           783000          
 2 Alaska   364900           95000            55200           
 3 Arizona  2883800          1346100          842000          
 4 Arkansas 1128800          600800           515200          
 5 Califor~ 17747300         8344800          3828500         
 6 Colorado 2852500          697300           549700          
 7 Connect~ 2030500          532000           475300          
 8 Delaware 473700           192700           141300          
 9 Distric~ 324300           174900           59900           
10 Florida  8023400          3190900          3108800         
# ... with 42 more rows, and 25 more variables: `2013__Non-Group` <chr>,
#   `2013__Other Public` <chr>, `2013__Total` <chr>,
#   `2013__Uninsured` <chr>, `2014__Employer` <chr>,
#   `2014__Medicaid` <chr>, `2014__Medicare` <chr>,
#   `2014__Non-Group` <chr>, `2014__Other Public` <chr>,
#   `2014__Total` <chr>, `2014__Uninsured` <chr>, `2015__Employer` <chr>,
#   `2015__Medicaid` <chr>, `2015__Medicare` <chr>,
#   `2015__Non-Group` <chr>, `2015__Other Public` <chr>,
#   `2015__Total` <chr>, `2015__Uninsured` <chr>, `2016__Employer` <chr>,
#   `2016__Medicaid` <chr>, `2016__Medicare` <chr>,
#   `2016__Non-Group` <chr>, `2016__Other Public` <chr>,
#   `2016__Total` <chr>, `2016__Uninsured` <chr>
\end{verbatim}

In the real world, analyzing data rarely involves data that can be
easily imported and ready for analysis. According to Wikipedia:

\begin{quote}
Data munging or data wrangling is loosely the process of manually
converting or mapping data from one ``raw'' form into another format
that allows for more convenient consumption of the data with the help of
semi-automated tools.
\end{quote}

As you may see in class or here from data scientists on Twitter, one of
the most time-consuming aspects of the data analysis process is ``data
wrangling''. This is also is a trendy term for \emph{cleaning up a messy
data set}.

R provides incredibly powerful and flexible language for data wrangling.
However, the syntax is somewhat hard to get used to. We will therefore
introducing a package that makes the syntax much more like the English
language. This package is \texttt{dplyr}.

\hypertarget{the-dplyr-r-package}{%
\subsection{\texorpdfstring{The \texttt{dplyr} R
package}{The dplyr R package}}\label{the-dplyr-r-package}}

\hypertarget{what-is-the-dplyr-r-package}{%
\subsubsection{\texorpdfstring{1. What is the \texttt{dplyr} R package
?}{1. What is the dplyr R package ?}}\label{what-is-the-dplyr-r-package}}

\href{http://cran.rstudio.com/web/packages/dplyr/vignettes/introduction.html}{\texttt{dplyr}}
is a powerful R-package to transform and summarize tabular data with
rows and columns.

The package contains a set of functions (or ``verbs'') to perform common
data manipulation operations such as filtering for rows, selecting
specific columns, re-ordering rows, adding new columns and summarizing
data.

In addition, \texttt{dplyr} contains a useful function to perform
another common task which is the is the ``split-apply-combine'' concept.
We will discuss that in a little bit.

\hypertarget{compare-dplyr-r-package-compare-with-base-functions-r}{%
\subsubsection{\texorpdfstring{2. Compare \texttt{dplyr} R package
compare with base functions
R}{2. Compare dplyr R package compare with base functions R}}\label{compare-dplyr-r-package-compare-with-base-functions-r}}

If you are familiar with R, you are probably familiar with base R
functions such as \texttt{split()}, \texttt{subset()}, \texttt{apply()},
\texttt{sapply()}, \texttt{lapply()}, \texttt{tapply()} and
\texttt{aggregate()}. Compared to base functions in R, the functions in
\texttt{dplyr} are easier to work with, are more consistent in the
syntax and are targeted for data analysis around data frames instead of
just vectors.

The important \texttt{dplyr} verbs to remember are:

\begin{longtable}[]{@{}ll@{}}
\toprule
\begin{minipage}[b]{0.03\columnwidth}\raggedright
\texttt{dplyr} verbs\strut
\end{minipage} & \begin{minipage}[b]{0.91\columnwidth}\raggedright
Description\strut
\end{minipage}\tabularnewline
\midrule
\endhead
\begin{minipage}[t]{0.03\columnwidth}\raggedright
\texttt{select()}\strut
\end{minipage} & \begin{minipage}[t]{0.91\columnwidth}\raggedright
select columns\strut
\end{minipage}\tabularnewline
\begin{minipage}[t]{0.03\columnwidth}\raggedright
\texttt{filter()}\strut
\end{minipage} & \begin{minipage}[t]{0.91\columnwidth}\raggedright
filter rows\strut
\end{minipage}\tabularnewline
\begin{minipage}[t]{0.03\columnwidth}\raggedright
\texttt{arrange()}\strut
\end{minipage} & \begin{minipage}[t]{0.91\columnwidth}\raggedright
re-order or arrange rows\strut
\end{minipage}\tabularnewline
\begin{minipage}[t]{0.03\columnwidth}\raggedright
\texttt{mutate()}\strut
\end{minipage} & \begin{minipage}[t]{0.91\columnwidth}\raggedright
create new columns\strut
\end{minipage}\tabularnewline
\begin{minipage}[t]{0.03\columnwidth}\raggedright
\texttt{summarize()}\strut
\end{minipage} & \begin{minipage}[t]{0.91\columnwidth}\raggedright
summarize values\strut
\end{minipage}\tabularnewline
\begin{minipage}[t]{0.03\columnwidth}\raggedright
\texttt{group\_by()}\strut
\end{minipage} & \begin{minipage}[t]{0.91\columnwidth}\raggedright
allows for group operations in the ``split-apply-combine'' concept\strut
\end{minipage}\tabularnewline
\bottomrule
\end{longtable}

\hypertarget{pipe-operator}{%
\subsubsection{3. Pipe operator:
\%\textgreater{}\%}\label{pipe-operator}}

Before we go any further, let's introduce the pipe operator:
\texttt{\%\textgreater{}\%}. In our \texttt{stocks} example, we briefly
saw this symbol. It is called the pipe operator. \texttt{dplyr} imports
this operator from another package (\texttt{magrittr})
\href{http://cran.r-project.org/web/packages/magrittr/vignettes/magrittr.html}{see
help file here}. This operator allows you to pipe the output from one
function to the input of another function. Instead of nesting functions
(reading from the inside to the outside), the idea of of piping is to
read the functions from left to right.

Now in \texttt{stocks} example, we pipe the \texttt{stocks} data frame
to the function that will gather multiple columns into key-value pairs.

\includegraphics{https://github.com/datasciencelabs/2016/raw/master/lectures/wrangling/pics/stocks-tidy.png}

\hypertarget{dplyr-verbs-in-action-separate-unite}{%
\paragraph{\texorpdfstring{\texttt{dplyr} verbs in action:
\texttt{separate()}, \texttt{unite()},
\ldots{}}{dplyr verbs in action: separate(), unite(), \ldots{}}}\label{dplyr-verbs-in-action-separate-unite}}

First, let's separate the \texttt{year\_type} column in the
\texttt{coverage} dataset to two columns: \texttt{year} and health
coverage \texttt{type}.

To do this, we will use the \texttt{separate()} function in the
\texttt{tidyr} package.

\textbf{Note}:

\begin{itemize}
\tightlist
\item
  \texttt{separate()} = separate one column into multiple columns
\item
  \texttt{unite()} = unite multiple columns into one
\end{itemize}

\hypertarget{learn-separate-and-unite-in-the-spending-dataset}{%
\paragraph{\texorpdfstring{Learn \texttt{separate()} and
\texttt{unite()} in the \texttt{spending}
dataset}{Learn separate() and unite() in the spending dataset}}\label{learn-separate-and-unite-in-the-spending-dataset}}

\begin{Shaded}
\begin{Highlighting}[]
\NormalTok{coverage }\OperatorTok\StringTok{ }
\StringTok{  }\KeywordTok{separate}\NormalTok{(year_type, }\DataTypeTok{sep=}\StringTok{"__"}\NormalTok{, }
           \DataTypeTok{into=}\KeywordTok{c}\NormalTok{(}\StringTok{"year"}\NormalTok{, }\StringTok{"type"}\NormalTok{))}
\end{Highlighting}
\end{Shaded}

\begin{verbatim}
# A tibble: 1,456 x 4
   Location             year  type     tot_coverage
   <chr>                <chr> <chr>    <chr>       
 1 United States        2013  Employer 155696900   
 2 Alabama              2013  Employer 2126500     
 3 Alaska               2013  Employer 364900      
 4 Arizona              2013  Employer 2883800     
 5 Arkansas             2013  Employer 1128800     
 6 California           2013  Employer 17747300    
 7 Colorado             2013  Employer 2852500     
 8 Connecticut          2013  Employer 2030500     
 9 Delaware             2013  Employer 473700      
10 District of Columbia 2013  Employer 324300      
# ... with 1,446 more rows
\end{verbatim}

We see that we now have two columns, except the \texttt{year} column was
converted to a character. If we look at the help file
\texttt{?separate}, we see we can use the \texttt{convert=TRUE} argument
to convert the character to an integer.

\begin{Shaded}
\begin{Highlighting}[]
\NormalTok{coverage <-}\StringTok{ }
\StringTok{  }\NormalTok{coverage }\OperatorTok\StringTok{ }
\StringTok{  }\KeywordTok{separate}\NormalTok{(year_type, }\DataTypeTok{sep=}\StringTok{"__"}\NormalTok{, }
           \DataTypeTok{into=}\KeywordTok{c}\NormalTok{(}\StringTok{"year"}\NormalTok{, }\StringTok{"type"}\NormalTok{), }
           \DataTypeTok{convert =} \OtherTok{TRUE}\NormalTok{)}
\NormalTok{coverage}
\end{Highlighting}
\end{Shaded}

\begin{verbatim}
# A tibble: 1,456 x 4
   Location              year type     tot_coverage
   <chr>                <int> <chr>    <chr>       
 1 United States         2013 Employer 155696900   
 2 Alabama               2013 Employer 2126500     
 3 Alaska                2013 Employer 364900      
 4 Arizona               2013 Employer 2883800     
 5 Arkansas              2013 Employer 1128800     
 6 California            2013 Employer 17747300    
 7 Colorado              2013 Employer 2852500     
 8 Connecticut           2013 Employer 2030500     
 9 Delaware              2013 Employer 473700      
10 District of Columbia  2013 Employer 324300      
# ... with 1,446 more rows
\end{verbatim}

Next, we see that the \texttt{tot\_coverage} column is also a character.
Gah!

Let's fix that. We can use the \texttt{mutate\_at()} function to do
this. We are asking R to take \texttt{tot\_coverage} column and convert
it to an integer and then replace the old column with the new converted
column

\begin{Shaded}
\begin{Highlighting}[]
\NormalTok{coverage <-}\StringTok{ }
\StringTok{  }\NormalTok{coverage }\OperatorTok\StringTok{ }
\StringTok{  }\KeywordTok{mutate_at}\NormalTok{(}\StringTok{"tot_coverage"}\NormalTok{, as.integer)}

\CommentTok{# Add the abbreviation of States}
\NormalTok{coverage}\OperatorTok{$}\NormalTok{abb <-}\StringTok{ }\NormalTok{state.abb[}\KeywordTok{match}\NormalTok{(coverage}\OperatorTok{$}\NormalTok{Location, state.name)]}
\NormalTok{coverage}\OperatorTok{$}\NormalTok{region <-}\StringTok{ }\NormalTok{state.region[}\KeywordTok{match}\NormalTok{(coverage}\OperatorTok{$}\NormalTok{Location, state.name)]}

\NormalTok{coverage}
\end{Highlighting}
\end{Shaded}

\begin{verbatim}
# A tibble: 1,456 x 6
   Location              year type     tot_coverage abb   region   
   <chr>                <int> <chr>           <int> <chr> <fct>    
 1 United States         2013 Employer    155696900 <NA>  <NA>     
 2 Alabama               2013 Employer      2126500 AL    South    
 3 Alaska                2013 Employer       364900 AK    West     
 4 Arizona               2013 Employer      2883800 AZ    West     
 5 Arkansas              2013 Employer      1128800 AR    South    
 6 California            2013 Employer     17747300 CA    West     
 7 Colorado              2013 Employer      2852500 CO    West     
 8 Connecticut           2013 Employer      2030500 CT    Northeast
 9 Delaware              2013 Employer       473700 DE    South    
10 District of Columbia  2013 Employer       324300 DC    South    
# ... with 1,446 more rows
\end{verbatim}

The \texttt{coverage} data looks good now. We see that there are
different \texttt{year}s and different \texttt{types} of healthcare
coverage.

Also, you may want to link the coverage data with our location
information.

\begin{Shaded}
\begin{Highlighting}[]
\CommentTok{# Add the abbreviation of States}
\NormalTok{coverage}\OperatorTok{$}\NormalTok{abb <-}\StringTok{ }\NormalTok{state.abb[}\KeywordTok{match}\NormalTok{(coverage}\OperatorTok{$}\NormalTok{Location, state.name)]}
\NormalTok{coverage}\OperatorTok{$}\NormalTok{region <-}\StringTok{ }\NormalTok{state.region[}\KeywordTok{match}\NormalTok{(coverage}\OperatorTok{$}\NormalTok{Location, state.name)]}

\NormalTok{coverage}
\end{Highlighting}
\end{Shaded}

\begin{verbatim}
# A tibble: 1,456 x 6
   Location              year type     tot_coverage abb   region   
   <chr>                <int> <chr>           <int> <chr> <fct>    
 1 United States         2013 Employer    155696900 <NA>  <NA>     
 2 Alabama               2013 Employer      2126500 AL    South    
 3 Alaska                2013 Employer       364900 AK    West     
 4 Arizona               2013 Employer      2883800 AZ    West     
 5 Arkansas              2013 Employer      1128800 AR    South    
 6 California            2013 Employer     17747300 CA    West     
 7 Colorado              2013 Employer      2852500 CO    West     
 8 Connecticut           2013 Employer      2030500 CT    Northeast
 9 Delaware              2013 Employer       473700 DE    South    
10 District of Columbia  2013 Employer       324300 DC    South    
# ... with 1,446 more rows
\end{verbatim}

\hypertarget{what-is-the-range-of-years-and-types-of-healthcare-in-the-coverage-dataset}{%
\paragraph{\texorpdfstring{(*) What is the range of years and types of
healthcare in the \texttt{coverage}
dataset?}{(*) What is the range of years and types of healthcare in the coverage dataset?}}\label{what-is-the-range-of-years-and-types-of-healthcare-in-the-coverage-dataset}}

\begin{Shaded}
\begin{Highlighting}[]
\KeywordTok{table}\NormalTok{(coverage}\OperatorTok{$}\NormalTok{type, coverage}\OperatorTok{$}\NormalTok{year)}
\end{Highlighting}
\end{Shaded}

\begin{verbatim}
              
               2013 2014 2015 2016
  Employer       52   52   52   52
  Medicaid       52   52   52   52
  Medicare       52   52   52   52
  Non-Group      52   52   52   52
  Other Public   52   52   52   52
  Total          52   52   52   52
  Uninsured      52   52   52   52
\end{verbatim}

\hypertarget{implement-separate-and-unite-in-the-spending-dataset}{%
\paragraph{\texorpdfstring{Implement \texttt{separate()} and
\texttt{unite()} in the \texttt{spending}
dataset}{Implement separate() and unite() in the spending dataset}}\label{implement-separate-and-unite-in-the-spending-dataset}}

Next, we will look at the \texttt{spending} data. We see the
\texttt{year} column has information that we do not want. We only care
about the year.

\begin{Shaded}
\begin{Highlighting}[]
\NormalTok{spending}
\end{Highlighting}
\end{Shaded}

\begin{verbatim}
# A tibble: 1,248 x 3
   Location             year                        tot_spending
   <chr>                <chr>                              <dbl>
 1 United States        1991__Total Health Spending       675896
 2 Alabama              1991__Total Health Spending        10393
 3 Alaska               1991__Total Health Spending         1458
 4 Arizona              1991__Total Health Spending         9269
 5 Arkansas             1991__Total Health Spending         5632
 6 California           1991__Total Health Spending        81438
 7 Colorado             1991__Total Health Spending         8460
 8 Connecticut          1991__Total Health Spending        10950
 9 Delaware             1991__Total Health Spending         1938
10 District of Columbia 1991__Total Health Spending         2800
# ... with 1,238 more rows
\end{verbatim}

Let's use the \texttt{separate()} function with \texttt{convert=TRUE} to
separate the \texttt{year} column into columns. Then, we introduce
another \texttt{dplyr} action verb: \texttt{select()}.

The two most basic functions are \texttt{select()} and \texttt{filter()}
which selects columns and filters rows, respectively.

\hypertarget{select-columns-using-select}{%
\subsubsection{\texorpdfstring{4. Select columns using
\texttt{select()}}{4. Select columns using select()}}\label{select-columns-using-select}}

In the \texttt{separate()} function, we create two new columns called
\texttt{year} and \texttt{name}. Then, we ask to return all the columns,
except \texttt{name}. To select all the columns \emph{except} a specific
column, use the ``-'' (subtraction) operator (also known as negative
indexing).

\begin{Shaded}
\begin{Highlighting}[]
\NormalTok{spending <-}\StringTok{ }
\StringTok{  }\NormalTok{spending }\OperatorTok\StringTok{ }
\StringTok{  }\KeywordTok{separate}\NormalTok{(year, }\DataTypeTok{sep=}\StringTok{"__"}\NormalTok{, }\DataTypeTok{into=}\KeywordTok{c}\NormalTok{(}\StringTok{"year"}\NormalTok{, }\StringTok{"name"}\NormalTok{), }\DataTypeTok{convert =} \OtherTok{TRUE}\NormalTok{) }\OperatorTok\StringTok{ }
\StringTok{  }\KeywordTok{select}\NormalTok{(}\OperatorTok{-}\NormalTok{name)}
\NormalTok{spending}
\end{Highlighting}
\end{Shaded}

\begin{verbatim}
# A tibble: 1,248 x 3
   Location              year tot_spending
   <chr>                <int>        <dbl>
 1 United States         1991       675896
 2 Alabama               1991        10393
 3 Alaska                1991         1458
 4 Arizona               1991         9269
 5 Arkansas              1991         5632
 6 California            1991        81438
 7 Colorado              1991         8460
 8 Connecticut           1991        10950
 9 Delaware              1991         1938
10 District of Columbia  1991         2800
# ... with 1,238 more rows
\end{verbatim}

The function \texttt{select()} is much more powerful though. To select a
range of columns by name, use the ``:'' (colon) operator

\begin{Shaded}
\begin{Highlighting}[]
\NormalTok{coverage }\OperatorTok\StringTok{ }
\StringTok{  }\KeywordTok{select}\NormalTok{(year}\OperatorTok{:}\NormalTok{type)}
\end{Highlighting}
\end{Shaded}

\begin{verbatim}
# A tibble: 1,456 x 2
    year type    
   <int> <chr>   
 1  2013 Employer
 2  2013 Employer
 3  2013 Employer
 4  2013 Employer
 5  2013 Employer
 6  2013 Employer
 7  2013 Employer
 8  2013 Employer
 9  2013 Employer
10  2013 Employer
# ... with 1,446 more rows
\end{verbatim}

To select all columns that start with the character string ``t'', use
the function \texttt{starts\_with()}

\begin{Shaded}
\begin{Highlighting}[]
\NormalTok{coverage }\OperatorTok\StringTok{ }
\StringTok{  }\KeywordTok{select}\NormalTok{(}\KeywordTok{starts_with}\NormalTok{(}\StringTok{"t"}\NormalTok{))}
\end{Highlighting}
\end{Shaded}

\begin{verbatim}
# A tibble: 1,456 x 2
   type     tot_coverage
   <chr>           <int>
 1 Employer    155696900
 2 Employer      2126500
 3 Employer       364900
 4 Employer      2883800
 5 Employer      1128800
 6 Employer     17747300
 7 Employer      2852500
 8 Employer      2030500
 9 Employer       473700
10 Employer       324300
# ... with 1,446 more rows
\end{verbatim}

Some additional options to select columns based on a specific criteria
include

\begin{enumerate}
\def\labelenumi{\arabic{enumi}.}
\tightlist
\item
  \texttt{ends\_with()} = Select columns that end with a character
  string
\item
  \texttt{contains()} = Select columns that contain a character string
\item
  \texttt{matches()} = Select columns that match a regular expression
\item
  \texttt{one\_of()} = Select columns names that are from a group of
  names
\end{enumerate}

\hypertarget{select-rows-using-filter}{%
\subsubsection{\texorpdfstring{5. Select rows using
\texttt{filter()}}{5. Select rows using filter()}}\label{select-rows-using-filter}}

Let's say we want to know how many peopled had health insurance coverage
in Maryland?

First, we can filter the rows for years in 2007.

\begin{Shaded}
\begin{Highlighting}[]
\NormalTok{coverage }\OperatorTok\StringTok{ }
\StringTok{  }\KeywordTok{filter}\NormalTok{(Location }\OperatorTok{==}\StringTok{ "Maryland"}\NormalTok{)}
\end{Highlighting}
\end{Shaded}

\begin{verbatim}
# A tibble: 28 x 6
   Location  year type         tot_coverage abb   region
   <chr>    <int> <chr>               <int> <chr> <fct> 
 1 Maryland  2013 Employer          3172400 MD    South 
 2 Maryland  2013 Non-Group          320800 MD    South 
 3 Maryland  2013 Medicaid           889800 MD    South 
 4 Maryland  2013 Medicare           751500 MD    South 
 5 Maryland  2013 Other Public       124400 MD    South 
 6 Maryland  2013 Uninsured          682000 MD    South 
 7 Maryland  2013 Total             5940900 MD    South 
 8 Maryland  2014 Employer          3558800 MD    South 
 9 Maryland  2014 Non-Group          361700 MD    South 
10 Maryland  2014 Medicaid           807900 MD    South 
# ... with 18 more rows
\end{verbatim}

\textbf{Note}: you can use the Boolean operators (e.g.
\texttt{\textgreater{}}, \texttt{\textless{}}, \texttt{\textgreater{}=},
\texttt{\textless{}=}, \texttt{!=}, \texttt{\%in\%}) to create logical
tests.

For example, if we wanted only years after 2014, we can add a second
criteria within \texttt{filter()}:

\begin{Shaded}
\begin{Highlighting}[]
\NormalTok{coverage }\OperatorTok\StringTok{ }
\StringTok{  }\KeywordTok{filter}\NormalTok{(Location }\OperatorTok{==}\StringTok{ "Maryland"}\NormalTok{, }
\NormalTok{         year }\OperatorTok{>}\StringTok{ }\DecValTok{2014}\NormalTok{)}
\end{Highlighting}
\end{Shaded}

\begin{verbatim}
# A tibble: 14 x 6
   Location  year type         tot_coverage abb   region
   <chr>    <int> <chr>               <int> <chr> <fct> 
 1 Maryland  2015 Employer          3431400 MD    South 
 2 Maryland  2015 Non-Group          371400 MD    South 
 3 Maryland  2015 Medicaid           856800 MD    South 
 4 Maryland  2015 Medicare           705500 MD    South 
 5 Maryland  2015 Other Public       141200 MD    South 
 6 Maryland  2015 Uninsured          394300 MD    South 
 7 Maryland  2015 Total             5900500 MD    South 
 8 Maryland  2016 Employer          3210600 MD    South 
 9 Maryland  2016 Non-Group          443000 MD    South 
10 Maryland  2016 Medicaid           926300 MD    South 
11 Maryland  2016 Medicare           827000 MD    South 
12 Maryland  2016 Other Public       153800 MD    South 
13 Maryland  2016 Uninsured          372100 MD    South 
14 Maryland  2016 Total             5932800 MD    South 
\end{verbatim}

\hypertarget{has-the-number-of-uninsured-has-increased-or-decreased-in-maryland-between-2013-and-2016}{%
\paragraph{(*) Has the number of uninsured has increased or decreased in
Maryland between 2013 and
2016?}\label{has-the-number-of-uninsured-has-increased-or-decreased-in-maryland-between-2013-and-2016}}

\begin{Shaded}
\begin{Highlighting}[]
\NormalTok{coverage }\OperatorTok\StringTok{ }
\StringTok{  }\KeywordTok{filter}\NormalTok{(Location }\OperatorTok{==}\StringTok{ "Maryland"}\NormalTok{, }
\NormalTok{         type }\OperatorTok{==}\StringTok{ "Uninsured"}\NormalTok{)}
\end{Highlighting}
\end{Shaded}

\begin{verbatim}
# A tibble: 4 x 6
  Location  year type      tot_coverage abb   region
  <chr>    <int> <chr>            <int> <chr> <fct> 
1 Maryland  2013 Uninsured       682000 MD    South 
2 Maryland  2014 Uninsured       343000 MD    South 
3 Maryland  2015 Uninsured       394300 MD    South 
4 Maryland  2016 Uninsured       372100 MD    South 
\end{verbatim}

What happened between 2013 and 2014?

\href{https://en.wikipedia.org/wiki/Patient_Protection_and_Affordable_Care_Act}{Probably
this is due to ACA}

\hypertarget{arrange-or-re-order-rows-using-arrange}{%
\subsubsection{\texorpdfstring{6. Arrange or re-order rows using
\texttt{arrange()}}{6. Arrange or re-order rows using arrange()}}\label{arrange-or-re-order-rows-using-arrange}}

Now, let's say we want to see the states ordered from lowest to highest
\texttt{tot\_coverage}.

To arrange (or re-order) rows by a particular column you'll use the
\texttt{arrange()} function:

\begin{Shaded}
\begin{Highlighting}[]
\NormalTok{coverage }\OperatorTok\StringTok{ }
\StringTok{    }\KeywordTok{arrange}\NormalTok{(tot_coverage)}
\end{Highlighting}
\end{Shaded}

\begin{verbatim}
# A tibble: 1,456 x 6
   Location       year type         tot_coverage abb   region       
   <chr>         <int> <chr>               <int> <chr> <fct>        
 1 Vermont        2013 Other Public         9900 VT    Northeast    
 2 Vermont        2014 Other Public         9900 VT    Northeast    
 3 Rhode Island   2013 Other Public        12100 RI    Northeast    
 4 Wyoming        2014 Other Public        13600 WY    West         
 5 Delaware       2013 Other Public        13800 DE    South        
 6 Vermont        2016 Other Public        14600 VT    Northeast    
 7 New Hampshire  2013 Other Public        15100 NH    Northeast    
 8 Wyoming        2016 Other Public        16400 WY    West         
 9 Vermont        2015 Other Public        16500 VT    Northeast    
10 North Dakota   2014 Other Public        17300 ND    North Central
# ... with 1,446 more rows
\end{verbatim}

\hypertarget{in-2016-what-were-the-top-three-states-with-the-largest-employer-type-of-healthcare-coverage}{%
\paragraph{\texorpdfstring{(*) In 2016, what were the top three states
with the largest \texttt{Employer} type of healthcare
coverage?}{(*) In 2016, what were the top three states with the largest Employer type of healthcare coverage?}}\label{in-2016-what-were-the-top-three-states-with-the-largest-employer-type-of-healthcare-coverage}}

\textbf{Hint}: use the \texttt{desc()} function inside of
\texttt{arrange()} to order rows in a descending order.

\begin{Shaded}
\begin{Highlighting}[]
\NormalTok{coverage }\OperatorTok\StringTok{ }
\StringTok{  }\KeywordTok{filter}\NormalTok{(Location }\OperatorTok{!=}\StringTok{ "United States"}\NormalTok{, year }\OperatorTok{==}\StringTok{ }\DecValTok{2016}\NormalTok{, type }\OperatorTok{==}\StringTok{ "Employer"}\NormalTok{) }\OperatorTok\StringTok{ }
\StringTok{  }\KeywordTok{arrange}\NormalTok{(}\KeywordTok{desc}\NormalTok{(tot_coverage)) }\OperatorTok\StringTok{ }
\StringTok{  }\KeywordTok{head}\NormalTok{(}\DataTypeTok{n=}\DecValTok{3}\NormalTok{)}
\end{Highlighting}
\end{Shaded}

\begin{verbatim}
# A tibble: 3 x 6
  Location    year type     tot_coverage abb   region   
  <chr>      <int> <chr>           <int> <chr> <fct>    
1 California  2016 Employer     18116200 CA    West     
2 Texas       2016 Employer     13607200 TX    South    
3 New York    2016 Employer      9767500 NY    Northeast
\end{verbatim}

\hypertarget{join-two-datasets-using-join}{%
\subsubsection{\texorpdfstring{7. Join two datasets using
\texttt{join()}}{7. Join two datasets using join()}}\label{join-two-datasets-using-join}}

Here, we're going to demonstrate how to join two datasets using series
of \texttt{join()} function, including \texttt{left\_join()},
\texttt{right\_join()}, \texttt{inner\_join()}, \ldots{}

Up until now, we have been working with three datasets \texttt{coverage}
and \texttt{spending} separately. Next, we will combine these together.

If we want to combine, say, \texttt{coverage} and \texttt{spending}
together, we have to decide a few things. Both share a \texttt{Location}
column and a \texttt{year} column. However, the range of \texttt{years}
is different between datasets.

\begin{Shaded}
\begin{Highlighting}[]
\KeywordTok{table}\NormalTok{(coverage}\OperatorTok{$}\NormalTok{year)}
\end{Highlighting}
\end{Shaded}

\begin{verbatim}

2013 2014 2015 2016 
 364  364  364  364 
\end{verbatim}

\begin{Shaded}
\begin{Highlighting}[]
\KeywordTok{table}\NormalTok{(spending}\OperatorTok{$}\NormalTok{year)}
\end{Highlighting}
\end{Shaded}

\begin{verbatim}

1991 1992 1993 1994 1995 1996 1997 1998 1999 2000 2001 2002 2003 2004 2005 
  52   52   52   52   52   52   52   52   52   52   52   52   52   52   52 
2006 2007 2008 2009 2010 2011 2012 2013 2014 
  52   52   52   52   52   52   52   52   52 
\end{verbatim}

Do we want a dataset with all the years available or only a portion of
it? Because there is spending information from 1991-2014, and coverage
information from 2013-2016.

\texttt{dplyr} has a list of \texttt{join} functions that are useful to
combine datasets. To read more about them, Jenny Bryan has created a
nice \href{http://stat545.com/bit001_dplyr-cheatsheet.html}{cheatsheet}.

If we look at the help file

\begin{Shaded}
\begin{Highlighting}[]
\NormalTok{?dplyr}\OperatorTok{::}\NormalTok{join}
\end{Highlighting}
\end{Shaded}

We see there are several options for us to pick from. Let's try one out.
We'll start with \texttt{left\_join()} and see what that does.

\begin{Shaded}
\begin{Highlighting}[]
\NormalTok{hc <-}\StringTok{ }\KeywordTok{left_join}\NormalTok{(coverage, spending, }\DataTypeTok{by =} \KeywordTok{c}\NormalTok{(}\StringTok{"Location"}\NormalTok{, }\StringTok{"year"}\NormalTok{))}
\KeywordTok{head}\NormalTok{(hc)}
\end{Highlighting}
\end{Shaded}

\begin{verbatim}
# A tibble: 6 x 7
  Location       year type     tot_coverage abb   region tot_spending
  <chr>         <int> <chr>           <int> <chr> <fct>         <dbl>
1 United States  2013 Employer    155696900 <NA>  <NA>        2435624
2 Alabama        2013 Employer      2126500 AL    South         33788
3 Alaska         2013 Employer       364900 AK    West           7684
4 Arizona        2013 Employer      2883800 AZ    West          41481
5 Arkansas       2013 Employer      1128800 AR    South         20500
6 California     2013 Employer     17747300 CA    West         278168
\end{verbatim}

\begin{Shaded}
\begin{Highlighting}[]
\KeywordTok{tail}\NormalTok{(hc)}
\end{Highlighting}
\end{Shaded}

\begin{verbatim}
# A tibble: 6 x 7
  Location       year type  tot_coverage abb   region        tot_spending
  <chr>         <int> <chr>        <int> <chr> <fct>                <dbl>
1 Vermont        2016 Total       622500 VT    Northeast               NA
2 Virginia       2016 Total      8175000 VA    South                   NA
3 Washington     2016 Total      7297300 WA    West                    NA
4 West Virginia  2016 Total      1814100 WV    South                   NA
5 Wisconsin      2016 Total      5766100 WI    North Central           NA
6 Wyoming        2016 Total       571700 WY    West                    NA
\end{verbatim}

What did it do? We see that the new \texttt{hc} dataset includes all the
years from 2013-2016 (as that is the range of years in
\texttt{coverage}), but because the \texttt{spending} dataset only goes
to 2014, the \texttt{tot\_spending} is reported as NA for years 2015 and
2016.

\hypertarget{what-about-a-right_join}{%
\paragraph{\texorpdfstring{What about a
\texttt{right\_join()}?}{What about a right\_join()?}}\label{what-about-a-right_join}}

\begin{Shaded}
\begin{Highlighting}[]
\NormalTok{hc <-}\StringTok{ }\KeywordTok{right_join}\NormalTok{(coverage, spending, }\DataTypeTok{by =} \KeywordTok{c}\NormalTok{(}\StringTok{"Location"}\NormalTok{, }\StringTok{"year"}\NormalTok{))}
\KeywordTok{head}\NormalTok{(hc)}
\end{Highlighting}
\end{Shaded}

\begin{verbatim}
# A tibble: 6 x 7
  Location       year type  tot_coverage abb   region tot_spending
  <chr>         <int> <chr>        <int> <chr> <fct>         <dbl>
1 United States  1991 <NA>            NA <NA>  <NA>         675896
2 Alabama        1991 <NA>            NA <NA>  <NA>          10393
3 Alaska         1991 <NA>            NA <NA>  <NA>           1458
4 Arizona        1991 <NA>            NA <NA>  <NA>           9269
5 Arkansas       1991 <NA>            NA <NA>  <NA>           5632
6 California     1991 <NA>            NA <NA>  <NA>          81438
\end{verbatim}

\begin{Shaded}
\begin{Highlighting}[]
\KeywordTok{tail}\NormalTok{(hc)}
\end{Highlighting}
\end{Shaded}

\begin{verbatim}
# A tibble: 6 x 7
  Location  year type         tot_coverage abb   region tot_spending
  <chr>    <int> <chr>               <int> <chr> <fct>         <dbl>
1 Wyoming   2014 Non-Group           31600 WY    West           4856
2 Wyoming   2014 Medicaid            54900 WY    West           4856
3 Wyoming   2014 Medicare            65600 WY    West           4856
4 Wyoming   2014 Other Public        13600 WY    West           4856
5 Wyoming   2014 Uninsured           58100 WY    West           4856
6 Wyoming   2014 Total              572000 WY    West           4856
\end{verbatim}

Here, we see every row in the spending dataset is there, but with NAs
for the years that there was no coverage data.

There is also a \texttt{full\_join()} and \texttt{inner\_join()}. If we
want the intersection of \texttt{years} from \texttt{coverage} and
\texttt{spending} (meaning only 2013 and 2014), we should use
\texttt{inner\_join()}.

\begin{Shaded}
\begin{Highlighting}[]
\NormalTok{hc <-}\StringTok{ }\KeywordTok{inner_join}\NormalTok{(coverage, spending, }\DataTypeTok{by =} \KeywordTok{c}\NormalTok{(}\StringTok{"Location"}\NormalTok{, }\StringTok{"year"}\NormalTok{))}
\KeywordTok{head}\NormalTok{(hc)}
\end{Highlighting}
\end{Shaded}

\begin{verbatim}
# A tibble: 6 x 7
  Location       year type     tot_coverage abb   region tot_spending
  <chr>         <int> <chr>           <int> <chr> <fct>         <dbl>
1 United States  2013 Employer    155696900 <NA>  <NA>        2435624
2 Alabama        2013 Employer      2126500 AL    South         33788
3 Alaska         2013 Employer       364900 AK    West           7684
4 Arizona        2013 Employer      2883800 AZ    West          41481
5 Arkansas       2013 Employer      1128800 AR    South         20500
6 California     2013 Employer     17747300 CA    West         278168
\end{verbatim}

\begin{Shaded}
\begin{Highlighting}[]
\KeywordTok{tail}\NormalTok{(hc)}
\end{Highlighting}
\end{Shaded}

\begin{verbatim}
# A tibble: 6 x 7
  Location       year type  tot_coverage abb   region        tot_spending
  <chr>         <int> <chr>        <int> <chr> <fct>                <dbl>
1 Vermont        2014 Total       617000 VT    Northeast             6389
2 Virginia       2014 Total      8258800 VA    South                62847
3 Washington     2014 Total      7085000 WA    West                 55819
4 West Virginia  2014 Total      1825500 WV    South                17491
5 Wisconsin      2014 Total      5747200 WI    North Central        50109
6 Wyoming        2014 Total       572000 WY    West                  4856
\end{verbatim}

Yes, that's what we want!

Next, if we are only interested in looking at US states, we can remove
the rows corresponding to the \texttt{Location\ ==\ "United\ States"}

\begin{Shaded}
\begin{Highlighting}[]
\NormalTok{hc <-}\StringTok{ }\NormalTok{hc }\OperatorTok\StringTok{ }
\StringTok{  }\KeywordTok{filter}\NormalTok{(Location }\OperatorTok{!=}\StringTok{ "United States"}\NormalTok{)}
\end{Highlighting}
\end{Shaded}

Another problem is that inside our \texttt{hc} dataset, we have seen
there are multiple \texttt{types} of healthcare coverage.

\begin{Shaded}
\begin{Highlighting}[]
\KeywordTok{table}\NormalTok{(hc}\OperatorTok{$}\NormalTok{type)}
\end{Highlighting}
\end{Shaded}

\begin{verbatim}

    Employer     Medicaid     Medicare    Non-Group Other Public 
         102          102          102          102          102 
       Total    Uninsured 
         102          102 
\end{verbatim}

The \texttt{total} type is not really a formal type of healthcare
coverage. It really represents just the total number of people in the
state. This is useful information and we can include it as a column
called \texttt{tot\_pop}. How can we do this?

Well, one way would be to use the \texttt{join} functions again in
\texttt{dplyr}.

\begin{Shaded}
\begin{Highlighting}[]
\NormalTok{pop <-}\StringTok{ }\NormalTok{hc }\OperatorTok\StringTok{ }
\StringTok{  }\KeywordTok{filter}\NormalTok{(type }\OperatorTok{==}\StringTok{ "Total"}\NormalTok{) }\OperatorTok\StringTok{ }
\StringTok{  }\KeywordTok{select}\NormalTok{(Location, year, tot_coverage)}
\NormalTok{pop}
\end{Highlighting}
\end{Shaded}

\begin{verbatim}
# A tibble: 102 x 3
   Location              year tot_coverage
   <chr>                <int>        <int>
 1 Alabama               2013      4763900
 2 Alaska                2013       702000
 3 Arizona               2013      6603100
 4 Arkansas              2013      2904800
 5 California            2013     38176400
 6 Colorado              2013      5297800
 7 Connecticut           2013      3578900
 8 Delaware              2013       909300
 9 District of Columbia  2013       652100
10 Florida               2013     19429000
# ... with 92 more rows
\end{verbatim}

\begin{Shaded}
\begin{Highlighting}[]
\NormalTok{hc <-}\StringTok{ }\NormalTok{hc }\OperatorTok\StringTok{ }
\StringTok{  }\KeywordTok{filter}\NormalTok{(type }\OperatorTok{!=}\StringTok{ "Total"}\NormalTok{) }\OperatorTok\StringTok{ }
\StringTok{  }\KeywordTok{left_join}\NormalTok{(pop, }\DataTypeTok{by =} \KeywordTok{c}\NormalTok{(}\StringTok{"Location"}\NormalTok{, }\StringTok{"year"}\NormalTok{)) }\OperatorTok\StringTok{ }
\StringTok{  }\KeywordTok{rename}\NormalTok{(}\DataTypeTok{tot_coverage =}\NormalTok{ tot_coverage.x, }\DataTypeTok{tot_pop =}\NormalTok{ tot_coverage.y)}
\NormalTok{hc}
\end{Highlighting}
\end{Shaded}

\begin{verbatim}
# A tibble: 612 x 8
   Location      year type   tot_coverage abb   region tot_spending tot_pop
   <chr>        <int> <chr>         <int> <chr> <fct>         <dbl>   <int>
 1 Alabama       2013 Emplo~      2126500 AL    South         33788  4.76e6
 2 Alaska        2013 Emplo~       364900 AK    West           7684  7.02e5
 3 Arizona       2013 Emplo~      2883800 AZ    West          41481  6.60e6
 4 Arkansas      2013 Emplo~      1128800 AR    South         20500  2.90e6
 5 California    2013 Emplo~     17747300 CA    West         278168  3.82e7
 6 Colorado      2013 Emplo~      2852500 CO    West          34090  5.30e6
 7 Connecticut   2013 Emplo~      2030500 CT    North~        34223  3.58e6
 8 Delaware      2013 Emplo~       473700 DE    South          9038  9.09e5
 9 District of~  2013 Emplo~       324300 DC    South          7443  6.52e5
10 Florida       2013 Emplo~      8023400 FL    South        150547  1.94e7
# ... with 602 more rows
\end{verbatim}

We can check to make sure that the \texttt{total} is no longer listed as
a \texttt{type} of healthcare coverage.

\begin{Shaded}
\begin{Highlighting}[]
\KeywordTok{table}\NormalTok{(hc}\OperatorTok{$}\NormalTok{type)}
\end{Highlighting}
\end{Shaded}

\begin{verbatim}

    Employer     Medicaid     Medicare    Non-Group Other Public 
         102          102          102          102          102 
   Uninsured 
         102 
\end{verbatim}

We are now ready to try answering our first question that we asked:

\begin{quote}
\begin{enumerate}
\def\labelenumi{\arabic{enumi}.}
\tightlist
\item
  Is there a relationship between healthcare coverage and healthcare
  spending in the United States?
\end{enumerate}
\end{quote}

Let's pick out the \texttt{type==Employer} and \texttt{year==2013}.

\begin{Shaded}
\begin{Highlighting}[]
\NormalTok{hc.employer}\FloatTok{.2013}\NormalTok{ <-}\StringTok{ }\NormalTok{hc }\OperatorTok
\StringTok{  }\KeywordTok{filter}\NormalTok{(type }\OperatorTok{==}\StringTok{ "Employer"}\NormalTok{, year }\OperatorTok{==}\StringTok{ "2013"}\NormalTok{)}
\KeywordTok{plot}\NormalTok{(hc.employer}\FloatTok{.2013}\OperatorTok{$}\NormalTok{tot_spending, }
\NormalTok{     hc.employer}\FloatTok{.2013}\OperatorTok{$}\NormalTok{tot_coverage, }\DataTypeTok{log =} \StringTok{"xy"}\NormalTok{, }
     \DataTypeTok{xlab =} \StringTok{"spending"}\NormalTok{, }\DataTypeTok{ylab =} \StringTok{"coverage"}\NormalTok{)}
\end{Highlighting}
\end{Shaded}

\includegraphics{ocs-healthexpenditure_files/figure-latex/unnamed-chunk-48-1.pdf}

We see there is a strong relationship. However, we also see that
healthcare coverage and spending is also strongly related to population
size

\begin{Shaded}
\begin{Highlighting}[]
\KeywordTok{par}\NormalTok{(}\DataTypeTok{mfrow=}\KeywordTok{c}\NormalTok{(}\DecValTok{1}\NormalTok{,}\DecValTok{2}\NormalTok{))}
\KeywordTok{plot}\NormalTok{(hc.employer}\FloatTok{.2013}\OperatorTok{$}\NormalTok{tot_pop, }
\NormalTok{     hc.employer}\FloatTok{.2013}\OperatorTok{$}\NormalTok{tot_coverage, }\DataTypeTok{log =} \StringTok{"xy"}\NormalTok{, }
     \DataTypeTok{xlab =} \StringTok{"population size"}\NormalTok{, }\DataTypeTok{ylab =} \StringTok{"coverage"}\NormalTok{)}
\KeywordTok{plot}\NormalTok{(hc.employer}\FloatTok{.2013}\OperatorTok{$}\NormalTok{tot_pop, }
\NormalTok{     hc.employer}\FloatTok{.2013}\OperatorTok{$}\NormalTok{tot_spending, }\DataTypeTok{log =} \StringTok{"xy"}\NormalTok{, }
     \DataTypeTok{xlab =} \StringTok{"population size"}\NormalTok{, }\DataTypeTok{ylab =} \StringTok{"spending"}\NormalTok{)}
\end{Highlighting}
\end{Shaded}

\includegraphics{ocs-healthexpenditure_files/figure-latex/unnamed-chunk-49-1.pdf}

This means we need to take into account the population size of each
state when we are comparing the healthcare coverage and spending.

\hypertarget{add-columns-using-mutate}{%
\subsubsection{\texorpdfstring{8. Add columns using
\texttt{mutate()}}{8. Add columns using mutate()}}\label{add-columns-using-mutate}}

Instead of the absolute number of people who are covered
(\texttt{tot\_coverage}), we will calculate the proportion of people who
are coverage in each state, year and type.

For this, we will use the \texttt{mutate()} function in \texttt{dplyr}.

\begin{Shaded}
\begin{Highlighting}[]
\NormalTok{hc <-}\StringTok{ }\NormalTok{hc }\OperatorTok\StringTok{ }
\StringTok{    }\KeywordTok{mutate}\NormalTok{(}\DataTypeTok{prop_coverage =}\NormalTok{ tot_coverage}\OperatorTok{/}\NormalTok{tot_pop) }
\NormalTok{hc}
\end{Highlighting}
\end{Shaded}

\begin{verbatim}
# A tibble: 612 x 9
   Location  year type  tot_coverage abb   region tot_spending tot_pop
   <chr>    <int> <chr>        <int> <chr> <fct>         <dbl>   <int>
 1 Alabama   2013 Empl~      2126500 AL    South         33788  4.76e6
 2 Alaska    2013 Empl~       364900 AK    West           7684  7.02e5
 3 Arizona   2013 Empl~      2883800 AZ    West          41481  6.60e6
 4 Arkansas  2013 Empl~      1128800 AR    South         20500  2.90e6
 5 Califor~  2013 Empl~     17747300 CA    West         278168  3.82e7
 6 Colorado  2013 Empl~      2852500 CO    West          34090  5.30e6
 7 Connect~  2013 Empl~      2030500 CT    North~        34223  3.58e6
 8 Delaware  2013 Empl~       473700 DE    South          9038  9.09e5
 9 Distric~  2013 Empl~       324300 DC    South          7443  6.52e5
10 Florida   2013 Empl~      8023400 FL    South        150547  1.94e7
# ... with 602 more rows, and 1 more variable: prop_coverage <dbl>
\end{verbatim}

We need to add another column to our dataset. We will add the spending
per capita (or spending per person) in dollars and name this column
\texttt{spending\_capita}.

\hypertarget{how-we-will-do-this}{%
\paragraph{How we will do this?}\label{how-we-will-do-this}}

The \texttt{tot\_spending} column is reported in millions (1e6).
Therefore, to calculate \texttt{spending\_capita} we will need to adjust
for this scaling factor to report it on the original scale (just
dollars) and then divide by \texttt{tot\_pop}.

\begin{Shaded}
\begin{Highlighting}[]
\NormalTok{hc <-}\StringTok{ }\NormalTok{hc }\OperatorTok\StringTok{ }
\StringTok{  }\KeywordTok{mutate}\NormalTok{(}\DataTypeTok{spending_capita =}\NormalTok{ (tot_spending}\OperatorTok{*}\FloatTok{1e6}\NormalTok{) }\OperatorTok{/}\StringTok{ }\NormalTok{tot_pop)}
\end{Highlighting}
\end{Shaded}

Now we are ready to go back to our first question.

\begin{quote}
\begin{enumerate}
\def\labelenumi{\arabic{enumi}.}
\tightlist
\item
  Is there a relationship between healthcare coverage and healthcare
  spending in the United States?
\end{enumerate}
\end{quote}

\begin{Shaded}
\begin{Highlighting}[]
\NormalTok{hc.employer}\FloatTok{.2013}\NormalTok{ <-}\StringTok{ }\NormalTok{hc }\OperatorTok
\StringTok{  }\KeywordTok{filter}\NormalTok{(type }\OperatorTok{==}\StringTok{ "Employer"}\NormalTok{, year }\OperatorTok{==}\StringTok{ "2013"}\NormalTok{)}
\KeywordTok{plot}\NormalTok{(hc.employer}\FloatTok{.2013}\OperatorTok{$}\NormalTok{spending_capita, }
\NormalTok{     hc.employer}\FloatTok{.2013}\OperatorTok{$}\NormalTok{prop_coverage, }\DataTypeTok{log =} \StringTok{"xy"}\NormalTok{, }
     \DataTypeTok{xlab =} \StringTok{"spending per capita"}\NormalTok{, }
     \DataTypeTok{ylab =} \StringTok{"proportion of Employer coverage"}\NormalTok{)}
\end{Highlighting}
\end{Shaded}

\includegraphics{ocs-healthexpenditure_files/figure-latex/unnamed-chunk-52-1.pdf}

Yes, it looks like there is a relationship for \texttt{Employer}
healthcare coverage in 2013.

We will continue to explore the other types of coverages later on. For
now, we get back to to learning more action verbs in \texttt{dplyr}.

Our second question that we were interested in was:

\begin{quote}
\begin{enumerate}
\def\labelenumi{\arabic{enumi}.}
\setcounter{enumi}{1}
\tightlist
\item
  Which US states spend the most and which spend the least on
  healthcare? How does the spending distribution change across
  geographic regions in the United States?
\end{enumerate}
\end{quote}

To answer these questions, we need to learn how to calculate summary
statistics in our data.

\hypertarget{create-summaries-of-columns-using-summarise}{%
\subsubsection{\texorpdfstring{9. Create summaries of columns using
\texttt{summarise()}}{9. Create summaries of columns using summarise()}}\label{create-summaries-of-columns-using-summarise}}

The \texttt{summarise()} function in \texttt{dplyr} will create summary
statistics for a given column in the data frame such as finding the max,
min, average. For example, to compute the average spending per capita,
we can apply the \texttt{mean()} function to the column
\texttt{spending\_captia} and call the summary value
\texttt{avg\_spending\_capita}.

\begin{Shaded}
\begin{Highlighting}[]
\NormalTok{hc }\OperatorTok\StringTok{ }
\StringTok{  }\KeywordTok{summarise}\NormalTok{(}\DataTypeTok{avg_spending_capita =} \KeywordTok{mean}\NormalTok{(spending_capita))}
\end{Highlighting}
\end{Shaded}

\begin{verbatim}
# A tibble: 1 x 1
  avg_spending_capita
                <dbl>
1               8246.
\end{verbatim}

There are many other summary statistics you could consider such
\texttt{sd()}, \texttt{min()}, \texttt{median()}, \texttt{mean()},
\texttt{sum()}, \texttt{n()} (returns the length of vector),
\texttt{first()} (returns first value in vector), \texttt{last()}
(returns last value in vector) and \texttt{n\_distinct()} (number of
distinct values in vector).

Also note, this is the average across all states, and all years. This is
not very informative.

If you recall, our question asked about \emph{which states} spent the
most, so we want an average spending per capita for each state.

For this, we need to introduce another function in \texttt{dplyr} called
\texttt{group\_by()}.

\hypertarget{group-operations-using-group_by}{%
\subsubsection{\texorpdfstring{10. Group operations using
\texttt{group\_by()}}{10. Group operations using group\_by()}}\label{group-operations-using-group_by}}

The \texttt{group\_by()} verb is and incredibly powerful function in
\texttt{dplyr}. As we mentioned before it's related to concept of
``split-apply-combine''.

In our example above, we want to split the data frame by some variable
(e.g. \texttt{Location}), apply a function to the individual data frames
(\texttt{mean}) and then combine the output back into a summary data
frame.

Let's see how that would look

\begin{Shaded}
\begin{Highlighting}[]
\NormalTok{hc }\OperatorTok\StringTok{ }
\StringTok{  }\KeywordTok{group_by}\NormalTok{(Location) }\OperatorTok
\StringTok{  }\KeywordTok{summarise}\NormalTok{(}\DataTypeTok{avg_spending_capita =} \KeywordTok{mean}\NormalTok{(spending_capita))}
\end{Highlighting}
\end{Shaded}

\begin{verbatim}
# A tibble: 51 x 2
   Location             avg_spending_capita
   <chr>                              <dbl>
 1 Alabama                            7244.
 2 Alaska                            11331.
 3 Arizona                            6397.
 4 Arkansas                           7324.
 5 California                         7416.
 6 Colorado                           6602.
 7 Connecticut                        9730.
 8 Delaware                          10127.
 9 District of Columbia              11698.
10 Florida                            7945.
# ... with 41 more rows
\end{verbatim}

That's better. Here we are averaging across the years 2013 and 2014.

\hypertarget{what-are-the-top-3-states-that-have-the-largest-average-spending-per-capita-what-about-the-top-3-states-with-the-smallest-average-spending-per-capita}{%
\paragraph{(*) What are the top 3 states that have the largest average
spending per capita? What about the top 3 states with the smallest
average spending per
capita?}\label{what-are-the-top-3-states-that-have-the-largest-average-spending-per-capita-what-about-the-top-3-states-with-the-smallest-average-spending-per-capita}}

\begin{Shaded}
\begin{Highlighting}[]
\CommentTok{# smallest }
\NormalTok{hc }\OperatorTok\StringTok{ }
\StringTok{  }\KeywordTok{group_by}\NormalTok{(Location) }\OperatorTok
\StringTok{  }\KeywordTok{summarise}\NormalTok{(}\DataTypeTok{avg_spending_capita =} \KeywordTok{mean}\NormalTok{(spending_capita)) }\OperatorTok\StringTok{ }
\StringTok{  }\KeywordTok{arrange}\NormalTok{(avg_spending_capita) }\OperatorTok\StringTok{ }
\StringTok{  }\KeywordTok{head}\NormalTok{(}\DataTypeTok{n=}\DecValTok{3}\NormalTok{)}
\end{Highlighting}
\end{Shaded}

\begin{verbatim}
# A tibble: 3 x 2
  Location avg_spending_capita
  <chr>                  <dbl>
1 Utah                   5842.
2 Arizona                6397.
3 Georgia                6513.
\end{verbatim}

\begin{Shaded}
\begin{Highlighting}[]
\CommentTok{# largest }
\NormalTok{hc }\OperatorTok\StringTok{ }
\StringTok{  }\KeywordTok{group_by}\NormalTok{(Location) }\OperatorTok
\StringTok{  }\KeywordTok{summarise}\NormalTok{(}\DataTypeTok{avg_spending_capita =} \KeywordTok{mean}\NormalTok{(spending_capita)) }\OperatorTok\StringTok{ }
\StringTok{  }\KeywordTok{arrange}\NormalTok{(}\KeywordTok{desc}\NormalTok{(avg_spending_capita)) }\OperatorTok\StringTok{ }
\StringTok{  }\KeywordTok{head}\NormalTok{(}\DataTypeTok{n=}\DecValTok{3}\NormalTok{)}
\end{Highlighting}
\end{Shaded}

\begin{verbatim}
# A tibble: 3 x 2
  Location             avg_spending_capita
  <chr>                              <dbl>
1 District of Columbia              11698.
2 Alaska                            11331.
3 Massachusetts                     10535.
\end{verbatim}

\hypertarget{how-does-the-spending-distribution-change-across-geographic-regions-in-the-united-states}{%
\paragraph{(*) How does the spending distribution change across
geographic regions in the United
States?}\label{how-does-the-spending-distribution-change-across-geographic-regions-in-the-united-states}}

\textbf{Hint}: Calculate the mean and standard deviation of spending per
capita for each geographic region in the US.

\begin{Shaded}
\begin{Highlighting}[]
\NormalTok{hc }\OperatorTok\StringTok{ }
\StringTok{  }\KeywordTok{group_by}\NormalTok{(region) }\OperatorTok
\StringTok{  }\KeywordTok{summarise}\NormalTok{(}\DataTypeTok{avg_spending_capita =} \KeywordTok{mean}\NormalTok{(spending_capita), }
            \DataTypeTok{sd_spending_capita =} \KeywordTok{sd}\NormalTok{(spending_capita))}
\end{Highlighting}
\end{Shaded}

\begin{verbatim}
# A tibble: 4 x 3
  region        avg_spending_capita sd_spending_capita
  <fct>                       <dbl>              <dbl>
1 North Central               8404.               541.
2 Northeast                   9592.               546.
3 South                       7994.              1273.
4 West                        7498.              1329.
\end{verbatim}

Another way to visualize distributions is to use boxplots.

Create four boxplots representing the spending per capita distribution
for each of the four regions using the \texttt{boxplot()} function in R.

\begin{Shaded}
\begin{Highlighting}[]
\KeywordTok{boxplot}\NormalTok{(hc}\OperatorTok{$}\NormalTok{spending_capita }\OperatorTok{~}\StringTok{ }\NormalTok{hc}\OperatorTok{$}\NormalTok{region)}
\end{Highlighting}
\end{Shaded}

\includegraphics{ocs-healthexpenditure_files/figure-latex/unnamed-chunk-57-1.pdf}

Now that we have our data in a \texttt{tidy} format, next, we will learn
about how to do this using the \texttt{ggplot2} R package in the
\texttt{tidyverse}.

\hypertarget{data-visualization}{%
\section{Data Visualization}\label{data-visualization}}

As you have already seen, there are many functions available in base R
that can create plots (e.g. \texttt{plot()}, \texttt{boxplot()}). Others
include: \texttt{hist()}, \texttt{qqplot()}, etc. These functions are
great because they come with a basic installation of R and can be quite
powerful when you need a quick visualization of something when you are
exploring data.

We are choosing to introduce \texttt{ggplot2} because, in our opinion,
it's one of the simplest ways for beginners to create relatively
complicated plots that are intuitive and aesthetically pleasing.

\hypertarget{the-ggplot2-r-package}{%
\subsection{\texorpdfstring{The \texttt{ggplot2} R
package}{The ggplot2 R package}}\label{the-ggplot2-r-package}}

The reasons \href{http://ggplot2.tidyverse.org}{\texttt{ggplot2}} is
generally intuitive for beginners is the use of
\href{http://vita.had.co.nz/papers/layered-grammar.html}{grammar of
graphics} or the \texttt{gg} in \texttt{ggplot2}. The idea is that you
can construct many sentences by learning just a few nouns, adjectives,
and verbs. There are specific ``words'' that we will need to learn and
once we do, you will be able to create (or ``write'') hundreds of
different plots.

The critical part to making graphics using \texttt{ggplot2} is the data
needs to be in a \emph{tidy} format. Given that we have just spend the
last two lectures learning about how to work with \emph{tidy} data, we
are primed to take advantage of all that \texttt{ggplot2} has to offer!

We will show how it's easy to pipe \emph{tidy} data (output) as input to
other functions that creates plots. This all works because we are
working within the \emph{tidyverse}.

\hypertarget{ggplot2-cheatsheet}{%
\paragraph{\texorpdfstring{\texttt{ggplot2}
cheatsheet}{ggplot2 cheatsheet}}\label{ggplot2-cheatsheet}}

The
\href{https://www.rstudio.com/wp-content/uploads/2015/03/ggplot2-cheatsheet.pdf}{cheatsheet}
looks like the following:

\includegraphics[width=12.62in]{./doc/ggplot2cheatsheet}

\hypertarget{what-is-the-ggplot-function}{%
\subsubsection{\texorpdfstring{1. What is the \texttt{ggplot()}
function?}{1. What is the ggplot() function?}}\label{what-is-the-ggplot-function}}

As explained by Hadley Wickham:

\begin{quote}
the grammar tells us that a statistical graphic is a mapping from data
to aesthetic attributes (colour, shape, size) of geometric objects
(points, lines, bars). The plot may also contain statistical
transformations of the data and is drawn on a specific coordinates
system.
\end{quote}

\hypertarget{ggplot2-terminology}{%
\paragraph{\texorpdfstring{\texttt{ggplot2}
Terminology}{ggplot2 Terminology}}\label{ggplot2-terminology}}

\begin{itemize}
\tightlist
\item
  \textbf{ggplot} - the main function where you specify the data set and
  variables to plot (this is where we define the \texttt{x} and
  \texttt{y} variable names)
\item
  \textbf{geoms} - geometric objects

  \begin{itemize}
  \tightlist
  \item
    e.g. \texttt{geom\_point()}, \texttt{geom\_bar()},
    \texttt{geom\_line()}, \texttt{geom\_histogram()}
  \end{itemize}
\item
  \textbf{aes} - aesthetics

  \begin{itemize}
  \tightlist
  \item
    shape, transparency, color, fill, linetype
  \end{itemize}
\item
  \textbf{scales} - define how your data will be plotted

  \begin{itemize}
  \tightlist
  \item
    continuous, discrete, log, etc
  \end{itemize}
\end{itemize}

There are three ways to initialize a \texttt{ggplot()} object.

An empty ggplot object

\begin{Shaded}
\begin{Highlighting}[]
\KeywordTok{library}\NormalTok{(ggplot2)}
\NormalTok{p <-}\StringTok{ }\KeywordTok{ggplot}\NormalTok{()         }
\end{Highlighting}
\end{Shaded}

A ggplot object associated with a dataset

\begin{Shaded}
\begin{Highlighting}[]
\NormalTok{p <-}\StringTok{ }\NormalTok{hc }\OperatorTok\StringTok{ }
\StringTok{      }\KeywordTok{filter}\NormalTok{(year}\OperatorTok{==}\DecValTok{2014}\NormalTok{) }\OperatorTok\StringTok{ }
\StringTok{      }\KeywordTok{ggplot}\NormalTok{()}
\end{Highlighting}
\end{Shaded}

or a ggplot object with a dataset and \texttt{x} and \texttt{y} defined

\begin{Shaded}
\begin{Highlighting}[]
\NormalTok{p <-}\StringTok{ }\NormalTok{hc }\OperatorTok\StringTok{ }
\StringTok{      }\KeywordTok{filter}\NormalTok{(year}\OperatorTok{==}\DecValTok{2014}\NormalTok{) }\OperatorTok\StringTok{ }
\StringTok{      }\KeywordTok{ggplot}\NormalTok{(}\KeywordTok{aes}\NormalTok{(}\DataTypeTok{x =}\NormalTok{ spending_capita, }\DataTypeTok{y =}\NormalTok{ prop_coverage))}
\end{Highlighting}
\end{Shaded}

\begin{Shaded}
\begin{Highlighting}[]
\NormalTok{p}
\end{Highlighting}
\end{Shaded}

\hypertarget{create-scatter-plots-using-geom_point}{%
\subsubsection{\texorpdfstring{2. Create scatter plots using
\texttt{geom\_point()}}{2. Create scatter plots using geom\_point()}}\label{create-scatter-plots-using-geom_point}}

The function \texttt{aes()} is an aesthetic mapping function inside the
\texttt{ggplot()} object. We use this function to specify plot
attributes (e.g. \texttt{x} and \texttt{y} variable names) that will not
change as we add more layers.

Anything that goes in the \texttt{ggplot()} object becomes a global
setting. From there, we use the \texttt{geom} objects to add more layers
to the base \texttt{ggplot()} object. These will define what we are
interested in illustrating using the data.

If you recall, our first question that we were interested in was

\begin{quote}
\begin{enumerate}
\def\labelenumi{\arabic{enumi}.}
\tightlist
\item
  Is there a relationship between healthcare coverage and healthcare
  spending in the United States?
\end{enumerate}
\end{quote}

Before, we were using base R to create something like this:

\begin{Shaded}
\begin{Highlighting}[]
\NormalTok{hc.employer}\FloatTok{.2013}\NormalTok{ <-}\StringTok{ }\NormalTok{hc }\OperatorTok\StringTok{ }
\StringTok{  }\KeywordTok{filter}\NormalTok{(type }\OperatorTok{==}\StringTok{ "Employer"}\NormalTok{, year }\OperatorTok{==}\StringTok{ "2013"}\NormalTok{)}

\KeywordTok{plot}\NormalTok{(hc.employer}\FloatTok{.2013}\OperatorTok{$}\NormalTok{spending_capita, }
\NormalTok{     hc.employer}\FloatTok{.2013}\OperatorTok{$}\NormalTok{prop_coverage, }
     \DataTypeTok{xlab =} \StringTok{"spending per capita"}\NormalTok{, }
     \DataTypeTok{ylab =} \StringTok{"coverage proportion"}\NormalTok{)}
\end{Highlighting}
\end{Shaded}

\includegraphics{ocs-healthexpenditure_files/figure-latex/unnamed-chunk-63-1.pdf}

Let's re-create this plot with \texttt{ggplot2} using the
\texttt{geom\_point()} geometry.

\begin{Shaded}
\begin{Highlighting}[]
\NormalTok{p <-}\StringTok{ }\NormalTok{hc }\OperatorTok
\StringTok{  }\KeywordTok{filter}\NormalTok{(type }\OperatorTok{==}\StringTok{ "Employer"}\NormalTok{, year }\OperatorTok{==}\StringTok{ "2013"}\NormalTok{) }\OperatorTok\StringTok{ }
\StringTok{  }\KeywordTok{ggplot}\NormalTok{(}\KeywordTok{aes}\NormalTok{(}\DataTypeTok{x =}\NormalTok{ spending_capita, }\DataTypeTok{y =}\NormalTok{ prop_coverage)) }
\NormalTok{p }\OperatorTok{+}\StringTok{ }\KeywordTok{geom_point}\NormalTok{() }\OperatorTok{+}\StringTok{ }
\StringTok{  }\KeywordTok{xlab}\NormalTok{(}\StringTok{"spending per capita"}\NormalTok{) }\OperatorTok{+}\StringTok{ }
\StringTok{  }\KeywordTok{ylab}\NormalTok{(}\StringTok{"coverage proportion"}\NormalTok{)}
\end{Highlighting}
\end{Shaded}

\includegraphics{ocs-healthexpenditure_files/figure-latex/unnamed-chunk-64-1.pdf}

We used the \texttt{xlab()} and \texttt{ylab()} functions in
\texttt{ggplot2} to specify the x-axis and y-axis labels.

\textbf{Note}, we do not have to assign (\texttt{\textless{}-}) the plot
to anything:

\begin{Shaded}
\begin{Highlighting}[]
\NormalTok{hc }\OperatorTok
\StringTok{  }\KeywordTok{filter}\NormalTok{(type }\OperatorTok{==}\StringTok{ "Employer"}\NormalTok{, year }\OperatorTok{==}\StringTok{ "2013"}\NormalTok{) }\OperatorTok\StringTok{ }
\StringTok{  }\KeywordTok{ggplot}\NormalTok{(}\KeywordTok{aes}\NormalTok{(}\DataTypeTok{x =}\NormalTok{ spending_capita, }\DataTypeTok{y =}\NormalTok{ prop_coverage)) }\OperatorTok{+}\StringTok{ }
\StringTok{  }\KeywordTok{geom_point}\NormalTok{() }\OperatorTok{+}\StringTok{ }
\StringTok{  }\KeywordTok{xlab}\NormalTok{(}\StringTok{"spending per capita"}\NormalTok{) }\OperatorTok{+}\StringTok{ }
\StringTok{  }\KeywordTok{ylab}\NormalTok{(}\StringTok{"coverage proportion"}\NormalTok{)}
\end{Highlighting}
\end{Shaded}

\includegraphics{ocs-healthexpenditure_files/figure-latex/unnamed-chunk-65-1.pdf}

It's also simple to fit a linear regression model and plot it on top of
scatter plot using the \texttt{geom\_smooth()} (or
\texttt{stat\_smooth()}) functions.

\begin{Shaded}
\begin{Highlighting}[]
\NormalTok{hc }\OperatorTok
\StringTok{  }\KeywordTok{filter}\NormalTok{(type }\OperatorTok{==}\StringTok{ "Employer"}\NormalTok{, year }\OperatorTok{==}\StringTok{ "2013"}\NormalTok{) }\OperatorTok\StringTok{ }
\StringTok{  }\KeywordTok{ggplot}\NormalTok{(}\KeywordTok{aes}\NormalTok{(}\DataTypeTok{x =}\NormalTok{ spending_capita, }\DataTypeTok{y =}\NormalTok{ prop_coverage)) }\OperatorTok{+}\StringTok{ }
\StringTok{  }\KeywordTok{geom_point}\NormalTok{() }\OperatorTok{+}\StringTok{ }
\StringTok{  }\KeywordTok{xlab}\NormalTok{(}\StringTok{"spending per capita"}\NormalTok{) }\OperatorTok{+}\StringTok{ }
\StringTok{  }\KeywordTok{ylab}\NormalTok{(}\StringTok{"coverage proportion"}\NormalTok{) }\OperatorTok{+}\StringTok{ }
\StringTok{  }\KeywordTok{geom_smooth}\NormalTok{(}\DataTypeTok{method =} \StringTok{"lm"}\NormalTok{, }\DataTypeTok{col =} \StringTok{"red"}\NormalTok{)}
\end{Highlighting}
\end{Shaded}

\includegraphics{ocs-healthexpenditure_files/figure-latex/unnamed-chunk-66-1.pdf}

The standard error bounds are computed and included in the plot.

It would be nice to know which state is represented by which state. For
this, we will introduce another \emph{geom} called
\texttt{geom\_text()}.

\hypertarget{add-layers-of-text-using-geom_text}{%
\subsubsection{\texorpdfstring{3. Add layers of text using
\texttt{geom\_text()}}{3. Add layers of text using geom\_text()}}\label{add-layers-of-text-using-geom_text}}

In our dataset, we have information about the abbreviation for each
state. We could add the abbreviations for each state next to the point
on the plot to assess which states have a higher or lower coverage for a
given amount of money they spend per capita.

\begin{Shaded}
\begin{Highlighting}[]
\NormalTok{hc }\OperatorTok\StringTok{ }
\StringTok{  }\KeywordTok{filter}\NormalTok{(type }\OperatorTok{==}\StringTok{ "Employer"}\NormalTok{, year }\OperatorTok{==}\StringTok{ "2013"}\NormalTok{) }\OperatorTok\StringTok{ }
\StringTok{  }\KeywordTok{ggplot}\NormalTok{(}\KeywordTok{aes}\NormalTok{(}\DataTypeTok{x =}\NormalTok{ spending_capita, }\DataTypeTok{y =}\NormalTok{ prop_coverage)) }\OperatorTok{+}\StringTok{ }
\StringTok{  }\KeywordTok{geom_point}\NormalTok{() }\OperatorTok{+}\StringTok{ }
\StringTok{  }\KeywordTok{xlab}\NormalTok{(}\StringTok{"spending per capita"}\NormalTok{) }\OperatorTok{+}\StringTok{ }
\StringTok{  }\KeywordTok{ylab}\NormalTok{(}\StringTok{"coverage proportion"}\NormalTok{) }\OperatorTok{+}\StringTok{ }
\StringTok{  }\KeywordTok{geom_smooth}\NormalTok{(}\DataTypeTok{method =} \StringTok{"lm"}\NormalTok{, }\DataTypeTok{col =} \StringTok{"red"}\NormalTok{) }\OperatorTok{+}\StringTok{ }
\StringTok{  }\KeywordTok{geom_text}\NormalTok{(}\KeywordTok{aes}\NormalTok{(}\DataTypeTok{label=}\NormalTok{abb))}
\end{Highlighting}
\end{Shaded}

\includegraphics{ocs-healthexpenditure_files/figure-latex/unnamed-chunk-67-1.pdf}

That is cool, but it would be even better if we could \emph{nudge} the
text over a bit. Let's look at the help file for \texttt{geom\_text()}:

\begin{Shaded}
\begin{Highlighting}[]
\NormalTok{?ggplot2}\OperatorTok{::}\NormalTok{geom_text}
\end{Highlighting}
\end{Shaded}

We see there is an argument called \texttt{nudge\_x} and
\texttt{nudge\_y}. We can use these to \emph{nudge} the text over a bit
so the text is not directly on top of the points.

\begin{Shaded}
\begin{Highlighting}[]
\NormalTok{hc }\OperatorTok\StringTok{ }
\StringTok{  }\KeywordTok{filter}\NormalTok{(type }\OperatorTok{==}\StringTok{ "Employer"}\NormalTok{, year }\OperatorTok{==}\StringTok{ "2013"}\NormalTok{) }\OperatorTok\StringTok{ }
\StringTok{  }\KeywordTok{ggplot}\NormalTok{(}\KeywordTok{aes}\NormalTok{(}\DataTypeTok{x =}\NormalTok{ spending_capita, }\DataTypeTok{y =}\NormalTok{ prop_coverage)) }\OperatorTok{+}\StringTok{ }
\StringTok{  }\KeywordTok{geom_point}\NormalTok{() }\OperatorTok{+}\StringTok{ }
\StringTok{  }\KeywordTok{xlab}\NormalTok{(}\StringTok{"spending per capita"}\NormalTok{) }\OperatorTok{+}\StringTok{ }
\StringTok{  }\KeywordTok{ylab}\NormalTok{(}\StringTok{"coverage proportion"}\NormalTok{) }\OperatorTok{+}\StringTok{ }
\StringTok{  }\KeywordTok{geom_smooth}\NormalTok{(}\DataTypeTok{method =} \StringTok{"lm"}\NormalTok{, }\DataTypeTok{col =} \StringTok{"red"}\NormalTok{) }\OperatorTok{+}\StringTok{ }
\StringTok{  }\KeywordTok{geom_text}\NormalTok{(}\KeywordTok{aes}\NormalTok{(}\DataTypeTok{label=}\NormalTok{abb), }\DataTypeTok{nudge_x =} \DecValTok{150}\NormalTok{)}
\end{Highlighting}
\end{Shaded}

\includegraphics{ocs-healthexpenditure_files/figure-latex/unnamed-chunk-69-1.pdf}

\hypertarget{color-each-point-or-state-by-what-region-they-are-from.}{%
\paragraph{(*) Color each point (or state) by what region they are
from.}\label{color-each-point-or-state-by-what-region-they-are-from.}}

\begin{Shaded}
\begin{Highlighting}[]
\CommentTok{## add your code here}

\NormalTok{hc }\OperatorTok\StringTok{ }
\StringTok{  }\KeywordTok{filter}\NormalTok{(type }\OperatorTok{==}\StringTok{ "Employer"}\NormalTok{, year }\OperatorTok{==}\StringTok{ "2013"}\NormalTok{) }\OperatorTok\StringTok{ }
\StringTok{  }\KeywordTok{ggplot}\NormalTok{(}\KeywordTok{aes}\NormalTok{(}\DataTypeTok{x =}\NormalTok{ spending_capita, }\DataTypeTok{y =}\NormalTok{ prop_coverage, }
             \DataTypeTok{color =}\NormalTok{ region)) }\OperatorTok{+}\StringTok{ }
\StringTok{  }\KeywordTok{geom_point}\NormalTok{() }\OperatorTok{+}\StringTok{ }
\StringTok{  }\KeywordTok{xlab}\NormalTok{(}\StringTok{"spending per capita"}\NormalTok{) }\OperatorTok{+}\StringTok{ }
\StringTok{  }\KeywordTok{ylab}\NormalTok{(}\StringTok{"coverage proportion"}\NormalTok{) }\OperatorTok{+}\StringTok{ }
\StringTok{  }\KeywordTok{geom_smooth}\NormalTok{(}\DataTypeTok{method =} \StringTok{"lm"}\NormalTok{, }\DataTypeTok{col =} \StringTok{"red"}\NormalTok{) }\OperatorTok{+}\StringTok{ }
\StringTok{  }\KeywordTok{geom_text}\NormalTok{(}\KeywordTok{aes}\NormalTok{(}\DataTypeTok{label=}\NormalTok{abb), }\DataTypeTok{nudge_x =} \DecValTok{150}\NormalTok{)}
\end{Highlighting}
\end{Shaded}

\includegraphics{ocs-healthexpenditure_files/figure-latex/unnamed-chunk-70-1.pdf}

\hypertarget{try-to-explore-the-package-ggrepel-and-check-if-you-can-improve-the-quality-of-visualization-using-the-geom_text_repel-from-ggrepel-instead-of-x_nudge-from-ggplot2.}{%
\paragraph{\texorpdfstring{(\emph{) Try to explore the package
\texttt{ggrepel}, and check if you can improve the quality of
visualization using the }geom\_text\_repel* from \texttt{ggrepel}
instead of \emph{x\_nudge} from
\texttt{ggplot2}.}{() Try to explore the package ggrepel, and check if you can improve the quality of visualization using the geom\_text\_repel* from ggrepel instead of x\_nudge from ggplot2.}}\label{try-to-explore-the-package-ggrepel-and-check-if-you-can-improve-the-quality-of-visualization-using-the-geom_text_repel-from-ggrepel-instead-of-x_nudge-from-ggplot2.}}

\begin{Shaded}
\begin{Highlighting}[]
\CommentTok{## add your code here  }

\KeywordTok{library}\NormalTok{(ggrepel)}

\NormalTok{hc }\OperatorTok\StringTok{ }
\StringTok{  }\KeywordTok{filter}\NormalTok{(type }\OperatorTok{==}\StringTok{ "Employer"}\NormalTok{, year }\OperatorTok{==}\StringTok{ "2013"}\NormalTok{) }\OperatorTok\StringTok{ }
\StringTok{  }\KeywordTok{ggplot}\NormalTok{(}\KeywordTok{aes}\NormalTok{(}\DataTypeTok{x =}\NormalTok{ spending_capita, }\DataTypeTok{y =}\NormalTok{ prop_coverage, }
             \DataTypeTok{color =}\NormalTok{ region)) }\OperatorTok{+}\StringTok{ }
\StringTok{  }\KeywordTok{geom_point}\NormalTok{() }\OperatorTok{+}\StringTok{ }
\StringTok{  }\KeywordTok{xlab}\NormalTok{(}\StringTok{"spending per capita"}\NormalTok{) }\OperatorTok{+}\StringTok{ }
\StringTok{  }\KeywordTok{ylab}\NormalTok{(}\StringTok{"coverage proportion"}\NormalTok{) }\OperatorTok{+}\StringTok{ }
\StringTok{  }\KeywordTok{geom_smooth}\NormalTok{(}\DataTypeTok{method =} \StringTok{"lm"}\NormalTok{, }\DataTypeTok{col =} \StringTok{"red"}\NormalTok{) }\OperatorTok{+}\StringTok{ }
\StringTok{  }\KeywordTok{geom_text_repel}\NormalTok{(}\KeywordTok{aes}\NormalTok{(}\DataTypeTok{label=}\NormalTok{abb))  }
\end{Highlighting}
\end{Shaded}

\includegraphics{ocs-healthexpenditure_files/figure-latex/unnamed-chunk-71-1.pdf}

\hypertarget{facet-across-a-variable-using-facet_wrap}{%
\subsubsection{\texorpdfstring{4. Facet across a variable using
\texttt{facet\_wrap}}{4. Facet across a variable using facet\_wrap}}\label{facet-across-a-variable-using-facet_wrap}}

Ok, getting back to our original question:

\begin{quote}
\begin{enumerate}
\def\labelenumi{\arabic{enumi}.}
\tightlist
\item
  Is there a relationship between healthcare coverage and healthcare
  spending in the United States?
\end{enumerate}
\end{quote}

We saw there was a positive relationship, but this was only for one type
of healthcare coverage (\texttt{Employer}) and one year. What about the
other types?

For this, we will introduce \texttt{facets}. The idea of \emph{faceting}
is to stratify the data by some variable and make the same plot for each
strata.

For example, if we wanted to \emph{facet} by the \texttt{type} variable,
we will add a layer to our \texttt{ggplot()} object using the
\texttt{facet\_grid()} or \texttt{facet\_wrap()} functions. The function
expects the row and column variables to be separated by a
\texttt{\textasciitilde{}}.

\begin{Shaded}
\begin{Highlighting}[]
\NormalTok{hc }\OperatorTok
\StringTok{  }\KeywordTok{filter}\NormalTok{(year }\OperatorTok{==}\StringTok{ "2013"}\NormalTok{) }\OperatorTok\StringTok{ }
\StringTok{  }\KeywordTok{ggplot}\NormalTok{(}\KeywordTok{aes}\NormalTok{(}\DataTypeTok{x =}\NormalTok{ spending_capita, }\DataTypeTok{y =}\NormalTok{ prop_coverage, }
             \DataTypeTok{color =}\NormalTok{ region)) }\OperatorTok{+}\StringTok{ }
\StringTok{  }\KeywordTok{geom_point}\NormalTok{() }\OperatorTok{+}\StringTok{ }
\StringTok{  }\KeywordTok{xlab}\NormalTok{(}\StringTok{"spending per capita"}\NormalTok{) }\OperatorTok{+}\StringTok{ }
\StringTok{  }\KeywordTok{ylab}\NormalTok{(}\StringTok{"coverage proportion"}\NormalTok{) }\OperatorTok{+}\StringTok{ }
\StringTok{  }\KeywordTok{geom_smooth}\NormalTok{(}\DataTypeTok{method =} \StringTok{"lm"}\NormalTok{, }\DataTypeTok{col =} \StringTok{"red"}\NormalTok{) }\OperatorTok{+}\StringTok{ }
\StringTok{  }\KeywordTok{geom_text_repel}\NormalTok{(}\KeywordTok{aes}\NormalTok{(}\DataTypeTok{label=}\NormalTok{abb)) }\OperatorTok{+}\StringTok{ }
\StringTok{  }\KeywordTok{facet_wrap}\NormalTok{(}\OperatorTok{~}\NormalTok{type)}
\end{Highlighting}
\end{Shaded}

\includegraphics{ocs-healthexpenditure_files/figure-latex/unnamed-chunk-72-1.pdf}

We see that the proportion of people covered have different scales in
the y-axis. Let's read the help file to see if there is some way to not
restrict the y-axis to be the same.

\begin{Shaded}
\begin{Highlighting}[]
\NormalTok{?ggplot2}\OperatorTok{::}\NormalTok{facet_grid}
\end{Highlighting}
\end{Shaded}

Yes, we see there is an argument called \texttt{scales} that can be
\texttt{free\_y}, (free columns), \texttt{free\_x} (free rows), and
\texttt{free} (both). Let's try \texttt{free\_y} and look at a different
year (\texttt{year=="2014"}):

\begin{Shaded}
\begin{Highlighting}[]
\NormalTok{hc }\OperatorTok
\StringTok{  }\KeywordTok{filter}\NormalTok{(year }\OperatorTok{==}\StringTok{ "2014"}\NormalTok{) }\OperatorTok\StringTok{ }
\StringTok{  }\KeywordTok{ggplot}\NormalTok{(}\KeywordTok{aes}\NormalTok{(}\DataTypeTok{x =}\NormalTok{ spending_capita, }\DataTypeTok{y =}\NormalTok{ prop_coverage, }
             \DataTypeTok{color =}\NormalTok{ region)) }\OperatorTok{+}\StringTok{ }
\StringTok{  }\KeywordTok{geom_point}\NormalTok{() }\OperatorTok{+}\StringTok{ }
\StringTok{  }\KeywordTok{xlab}\NormalTok{(}\StringTok{"spending per capita"}\NormalTok{) }\OperatorTok{+}\StringTok{ }
\StringTok{  }\KeywordTok{ylab}\NormalTok{(}\StringTok{"coverage proportion"}\NormalTok{) }\OperatorTok{+}\StringTok{ }
\StringTok{  }\KeywordTok{geom_smooth}\NormalTok{(}\DataTypeTok{method =} \StringTok{"lm"}\NormalTok{, }\DataTypeTok{col =} \StringTok{"red"}\NormalTok{) }\OperatorTok{+}\StringTok{ }
\StringTok{  }\KeywordTok{geom_text_repel}\NormalTok{(}\KeywordTok{aes}\NormalTok{(}\DataTypeTok{label=}\NormalTok{abb)) }\OperatorTok{+}\StringTok{ }
\StringTok{  }\KeywordTok{facet_wrap}\NormalTok{(}\OperatorTok{~}\NormalTok{type, }\DataTypeTok{scales=}\StringTok{"free_y"}\NormalTok{)}
\end{Highlighting}
\end{Shaded}

\includegraphics{ocs-healthexpenditure_files/figure-latex/unnamed-chunk-74-1.pdf}

Given we know \texttt{Other\ Public} refers to the military or Veterans
Administration, we can see states like HI, VA, NV have a larger
proportion of military or VA \texttt{Other\ Public} type coverage. While
a state like AK has a similar proportion of \texttt{Other\ Public}
coverage, it has a much larger spending per capita.

We also see a negative relationship with the \texttt{Uninsured} type.
The more states spend, the less uninsured people in the state.

\hypertarget{create-boxplots-using-geom_boxplot}{%
\subsubsection{\texorpdfstring{5. Create boxplots using
\texttt{geom\_boxplot()}}{5. Create boxplots using geom\_boxplot()}}\label{create-boxplots-using-geom_boxplot}}

Next, let's revisit the second question.

\begin{quote}
\begin{enumerate}
\def\labelenumi{\arabic{enumi}.}
\setcounter{enumi}{1}
\tightlist
\item
  Which US states spend the most and which spend the least on
  healthcare? How does the spending distribution change across
  geographic regions in the United States?
\end{enumerate}
\end{quote}

Let's try making a boxplot with \texttt{ggplot2}. If you recall, the way
to do this in base R was:

\begin{Shaded}
\begin{Highlighting}[]
\KeywordTok{boxplot}\NormalTok{(hc}\OperatorTok{$}\NormalTok{spending_capita }\OperatorTok{~}\StringTok{ }\NormalTok{hc}\OperatorTok{$}\NormalTok{region)}
\end{Highlighting}
\end{Shaded}

\includegraphics{ocs-healthexpenditure_files/figure-latex/unnamed-chunk-75-1.pdf}

Now, we introduce the \texttt{geom\_boxplot()} function. Note, we needed
to tell \texttt{ggplot2} what needs to be along the x and y axis in
\texttt{aes()}.

\begin{Shaded}
\begin{Highlighting}[]
\NormalTok{hc }\OperatorTok\StringTok{ }
\StringTok{  }\KeywordTok{ggplot}\NormalTok{(}\KeywordTok{aes}\NormalTok{(}\DataTypeTok{x =}\NormalTok{ region, }\DataTypeTok{y =}\NormalTok{ spending_capita)) }\OperatorTok{+}\StringTok{ }
\StringTok{  }\KeywordTok{geom_boxplot}\NormalTok{()}
\end{Highlighting}
\end{Shaded}

\includegraphics{ocs-healthexpenditure_files/figure-latex/unnamed-chunk-76-1.pdf}

\hypertarget{facet-by-two-variables-using-facet_grid}{%
\subsubsection{\texorpdfstring{6. Facet by two variables using
\texttt{facet\_grid}}{6. Facet by two variables using facet\_grid}}\label{facet-by-two-variables-using-facet_grid}}

\begin{quote}
\begin{enumerate}
\def\labelenumi{\arabic{enumi}.}
\setcounter{enumi}{2}
\tightlist
\item
  Does the relationship between healthcare coverage and healthcare
  spending in the United States change from 2013 to 2014?
\end{enumerate}
\end{quote}

Let's try faceting by both \texttt{year} and \texttt{type}. Note that we
can facet by rows putting a column name before the
\texttt{\textasciitilde{}} and facet by columns putting a column name
after the \texttt{\textasciitilde{}}. We are also using
\texttt{facet\_grid()} instead of \texttt{facet\_wrap()}.

\begin{Shaded}
\begin{Highlighting}[]
\NormalTok{p <-}\StringTok{ }\NormalTok{hc }\OperatorTok
\StringTok{  }\KeywordTok{ggplot}\NormalTok{(}\KeywordTok{aes}\NormalTok{(}\DataTypeTok{x =}\NormalTok{ spending_capita, }\DataTypeTok{y =}\NormalTok{ prop_coverage, }
             \DataTypeTok{color =}\NormalTok{ region)) }\OperatorTok{+}\StringTok{ }
\StringTok{  }\KeywordTok{geom_point}\NormalTok{() }\OperatorTok{+}\StringTok{ }
\StringTok{  }\KeywordTok{xlab}\NormalTok{(}\StringTok{"spending per capita"}\NormalTok{) }\OperatorTok{+}\StringTok{ }
\StringTok{  }\KeywordTok{ylab}\NormalTok{(}\StringTok{"coverage proportion"}\NormalTok{) }\OperatorTok{+}\StringTok{ }
\StringTok{  }\KeywordTok{geom_smooth}\NormalTok{(}\DataTypeTok{method =} \StringTok{"lm"}\NormalTok{, }\DataTypeTok{col =} \StringTok{"red"}\NormalTok{) }\OperatorTok{+}\StringTok{ }
\StringTok{  }\KeywordTok{geom_text_repel}\NormalTok{(}\KeywordTok{aes}\NormalTok{(}\DataTypeTok{label=}\NormalTok{abb)) }

\NormalTok{p }\OperatorTok{+}\StringTok{ }\KeywordTok{facet_grid}\NormalTok{(year}\OperatorTok{~}\NormalTok{type, }\DataTypeTok{scales=}\StringTok{"free"}\NormalTok{)}
\end{Highlighting}
\end{Shaded}

\includegraphics{ocs-healthexpenditure_files/figure-latex/unnamed-chunk-77-1.pdf}

\begin{Shaded}
\begin{Highlighting}[]
\NormalTok{p }\OperatorTok{+}\StringTok{ }\KeywordTok{facet_grid}\NormalTok{(year}\OperatorTok{~}\NormalTok{type)}
\end{Highlighting}
\end{Shaded}

\includegraphics{ocs-healthexpenditure_files/figure-latex/unnamed-chunk-77-2.pdf}

\hypertarget{summary}{%
\section{Summary}\label{summary}}

The total healthcare expenditure is associated with the population. To
make a fair comparison, we create ``healthcare expenditure per capita.''
Further, the exploratory analysis via data visualization showed higher
spending in healthcare per capita is positively associated with higher
employer coverage proportion and is negatively associated with the
proportion of uninsured population across the States.


\end{document}
